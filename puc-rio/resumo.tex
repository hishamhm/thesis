
Muitas aplicações são tornadas programáveis para usuários finais avançados
adicionando recursos como scripting e macros. Outras aplicações dão
a uma linguagem de programação um papel central na sua interface com
o usuário. Esse é o caso, por exemplo, da linguagem de fórmulas de
planilhas de cálculo. Enquanto a área de scripting se beneficiou dos
avanços das pesquisas em linguagens de programação, produzindo linguagens
maduras e reusáveis, o estado das linguagens em nível de interface
não teve o mesmo grau de desenvolvimento. Argumentamos que um melhor
entendimento desta classe de linguagens se faz necessário. Neste trabalho,
modelamos semânticas de linguagens de usuário final existentes, em
três diferentes domínios: multimídia, planilhas e engenharia. Nosso
foco é em linguagens de dataflow, um paradigma representativo em aplicações
programáveis por usuários finais. Com base nessa análise, temos como
objetivo prover um melhor entendimento do design de linguagens de
dataflow no contexto de programação de usuários finais e propor linhas-guia
para o projeto de linguagens de nível de interface baseadas neste
paradigma para aplicações programáveis. 

