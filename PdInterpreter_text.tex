%BEGIN LYX TEXT

This implementation uses only standard modules included in the Haskell Platform:

\begin{hscode}\SaveRestoreHook
\column{B}{@{}>{\hspre}l<{\hspost}@{}}%
\column{E}{@{}>{\hspre}l<{\hspost}@{}}%
\>[B]{}\mathbf{import}\;\Conid{\Conid{Data}.Sequence}\;(\Conid{Seq},\Varid{fromList},\Varid{index},\Varid{update},\Varid{foldlWithIndex}){}\<[E]%
\\
\>[B]{}\mathbf{import}\;\Varid{qualified}\;\Conid{\Conid{Data}.Sequence}\;\Varid{as}\;\Conid{Seq}\;(\Varid{length}){}\<[E]%
\\
\>[B]{}\mathbf{import}\;\Conid{\Conid{Data}.Foldable}\;(\Varid{foldl'},\Varid{toList}){}\<[E]%
\\
\>[B]{}\mathbf{import}\;\Conid{\Conid{Data}.List}\;(\Varid{sort},\Varid{intercalate},\Varid{find}){}\<[E]%
\\
\>[B]{}\mathbf{import}\;\Conid{\Conid{Text}.Printf}{}\<[E]%
\\
\>[B]{}\mathbf{import}\;\Conid{\Conid{Data}.Fixed}{}\<[E]%
\\
\>[B]{}\mathbf{import}\;\Conid{\Conid{Data}.\Conid{Binary}.Put}{}\<[E]%
\\
\>[B]{}\mathbf{import}\;\Varid{qualified}\;\Conid{\Conid{Data}.\Conid{ByteString}.Lazy}\;\Varid{as}\;\Conid{ByteString}{}\<[E]%
\\
\>[B]{}\mathbf{import}\;\Conid{\Conid{Control}.Monad}{}\<[E]%
\\
\>[B]{}\mathbf{import}\;\Conid{\Conid{Debug}.Trace}{}\<[E]%
\ColumnHook
\end{hscode}\resethooks

\subsection{Representation of programs}

A Pure Data program (called a ``patch'') is represented with the
\ensuremath{\Conid{PdPatch}} data type in our model, which contains a sequence of nodes, a
sequence of connections between nodes, and the pre-computed topological sort
of audio connections (stored as a list of integer indices).

\begin{hscode}\SaveRestoreHook
\column{B}{@{}>{\hspre}l<{\hspost}@{}}%
\column{17}{@{}>{\hspre}l<{\hspost}@{}}%
\column{20}{@{}>{\hspre}l<{\hspost}@{}}%
\column{30}{@{}>{\hspre}l<{\hspost}@{}}%
\column{E}{@{}>{\hspre}l<{\hspost}@{}}%
\>[B]{}\mathbf{data}\;\Conid{PdPatch}\mathrel{=}{}\<[17]%
\>[17]{}\Conid{PdPatch}\;\{\mskip1.5mu {}\<[E]%
\\
\>[17]{}\hsindent{3}{}\<[20]%
\>[20]{}\Varid{pNodes}{}\<[30]%
\>[30]{}\mathbin{::}\Conid{Seq}\;\Conid{PdNode},{}\<[E]%
\\
\>[17]{}\hsindent{3}{}\<[20]%
\>[20]{}\Varid{pConns}{}\<[30]%
\>[30]{}\mathbin{::}\Conid{Seq}\;(\cdot \rhd \cdot ),{}\<[E]%
\\
\>[17]{}\hsindent{3}{}\<[20]%
\>[20]{}\Varid{pDspSort}{}\<[30]%
\>[30]{}\mathbin{::}[\mskip1.5mu \Conid{Int}\mskip1.5mu]{}\<[E]%
\\
\>[17]{}\mskip1.5mu\}{}\<[E]%
\ColumnHook
\end{hscode}\resethooks

The primitive values in Pure Data are called ``atoms'': they can be numbers,
strings (called ``symbols'') or opaque pointers. Opaque pointers are used by
graphical objects only, so those will be omitted here.

\begin{hscode}\SaveRestoreHook
\column{B}{@{}>{\hspre}l<{\hspost}@{}}%
\column{4}{@{}>{\hspre}l<{\hspost}@{}}%
\column{14}{@{}>{\hspre}c<{\hspost}@{}}%
\column{14E}{@{}l@{}}%
\column{17}{@{}>{\hspre}l<{\hspost}@{}}%
\column{25}{@{}>{\hspre}l<{\hspost}@{}}%
\column{29}{@{}>{\hspre}l<{\hspost}@{}}%
\column{E}{@{}>{\hspre}l<{\hspost}@{}}%
\>[B]{}\mathbf{data}\;\Conid{PdAtom}{}\<[14]%
\>[14]{}\mathrel{=}{}\<[14E]%
\>[17]{}\Conid{PdFloat}\;{}\<[29]%
\>[29]{}\Conid{Double}{}\<[E]%
\\
\>[14]{}\mid {}\<[14E]%
\>[17]{}\Conid{PdSymbol}\;{}\<[29]%
\>[29]{}\Conid{String}{}\<[E]%
\\
\>[B]{}\hsindent{4}{}\<[4]%
\>[4]{}\mathbf{deriving}\;(\Conid{Eq},\Conid{Ord}){}\<[E]%
\\[\blanklineskip]%
\>[B]{}\mathbf{instance}\;\Conid{Show}\;\Conid{PdAtom}\;\mathbf{where}{}\<[E]%
\\
\>[B]{}\hsindent{4}{}\<[4]%
\>[4]{}\Varid{show}\;(\Conid{PdFloat}\;\Varid{f}){}\<[25]%
\>[25]{}\mathrel{=}\Varid{show}\;\Varid{f}{}\<[E]%
\\
\>[B]{}\hsindent{4}{}\<[4]%
\>[4]{}\Varid{show}\;(\Conid{PdSymbol}\;\Varid{s}){}\<[25]%
\>[25]{}\mathrel{=}\Varid{s}{}\<[E]%
\ColumnHook
\end{hscode}\resethooks

Nodes may be objects, atom boxes or message boxes. In Pure Data, objects are
initialized via ``creation arguments'': a string of arguments, represented
here as a list of atoms. We also store in an object its number of inlets and
outlets. Atom boxes and message boxes always have one inlet and one outlet
each.

\begin{hscode}\SaveRestoreHook
\column{B}{@{}>{\hspre}l<{\hspost}@{}}%
\column{4}{@{}>{\hspre}l<{\hspost}@{}}%
\column{14}{@{}>{\hspre}c<{\hspost}@{}}%
\column{14E}{@{}l@{}}%
\column{17}{@{}>{\hspre}l<{\hspost}@{}}%
\column{28}{@{}>{\hspre}l<{\hspost}@{}}%
\column{E}{@{}>{\hspre}l<{\hspost}@{}}%
\>[B]{}\mathbf{data}\;\Conid{PdNode}{}\<[14]%
\>[14]{}\mathrel{=}{}\<[14E]%
\>[17]{}\Conid{PdObj}\;{}\<[28]%
\>[28]{}[\mskip1.5mu \Conid{PdAtom}\mskip1.5mu]\;\Conid{Int}\;\Conid{Int}{}\<[E]%
\\
\>[14]{}\mid {}\<[14E]%
\>[17]{}\Conid{PdAtomBox}\;{}\<[28]%
\>[28]{}\Conid{PdAtom}{}\<[E]%
\\
\>[14]{}\mid {}\<[14E]%
\>[17]{}\Conid{PdMsgBox}\;{}\<[28]%
\>[28]{}[\mskip1.5mu \Conid{PdCmd}\mskip1.5mu]{}\<[E]%
\\
\>[B]{}\hsindent{4}{}\<[4]%
\>[4]{}\mathbf{deriving}\;\Conid{Show}{}\<[E]%
\ColumnHook
\end{hscode}\resethooks

Message boxes contain commands written in the textual sub-language of Pure
Data. Here, we represent commands not as a string, but in parsed form,
consisting of a receiver and a list of tokens (which may be literal values or
numbered references to inlet data (written \texttt{\$\emph{n}} in the textual language).
Note that a single message box may contain a number of messages.

\begin{hscode}\SaveRestoreHook
\column{B}{@{}>{\hspre}l<{\hspost}@{}}%
\column{4}{@{}>{\hspre}l<{\hspost}@{}}%
\column{15}{@{}>{\hspre}c<{\hspost}@{}}%
\column{15E}{@{}l@{}}%
\column{18}{@{}>{\hspre}l<{\hspost}@{}}%
\column{21}{@{}>{\hspre}l<{\hspost}@{}}%
\column{29}{@{}>{\hspre}l<{\hspost}@{}}%
\column{E}{@{}>{\hspre}l<{\hspost}@{}}%
\>[B]{}\mathbf{data}\;\Conid{PdCmd}\mathrel{=}\Conid{PdCmd}\;\Conid{PdReceiver}\;[\mskip1.5mu \Conid{PdToken}\mskip1.5mu]{}\<[E]%
\\
\>[B]{}\hsindent{4}{}\<[4]%
\>[4]{}\mathbf{deriving}\;\Conid{Show}{}\<[E]%
\\[\blanklineskip]%
\>[B]{}\mathbf{data}\;\Conid{PdReceiver}{}\<[18]%
\>[18]{}\mathrel{=}{}\<[21]%
\>[21]{}\Conid{PdToOutlet}{}\<[E]%
\\
\>[18]{}\mid {}\<[21]%
\>[21]{}\Conid{PdReceiver}\;\Conid{String}{}\<[E]%
\\
\>[18]{}\mid {}\<[21]%
\>[21]{}\Conid{PdRDollar}\;\Conid{Int}{}\<[E]%
\\
\>[18]{}\mid {}\<[21]%
\>[21]{}\Conid{PdReceiverErr}{}\<[E]%
\\
\>[B]{}\hsindent{4}{}\<[4]%
\>[4]{}\mathbf{deriving}\;\Conid{Show}{}\<[E]%
\\[\blanklineskip]%
\>[B]{}\mathbf{data}\;\Conid{PdToken}{}\<[15]%
\>[15]{}\mathrel{=}{}\<[15E]%
\>[18]{}\Conid{PdTDollar}\;{}\<[29]%
\>[29]{}\Conid{Int}{}\<[E]%
\\
\>[15]{}\mid {}\<[15E]%
\>[18]{}\Conid{PdTAtom}\;{}\<[29]%
\>[29]{}\Conid{PdAtom}{}\<[E]%
\\
\>[B]{}\hsindent{4}{}\<[4]%
\>[4]{}\mathbf{deriving}\;\Conid{Show}{}\<[E]%
\ColumnHook
\end{hscode}\resethooks

Finally, we represent the connections of the graph as a sequence of adjacency
pairs, where each pair is represented as a \ensuremath{(\cdot \rhd \cdot )} value, itself
composed of two pairs: the node index and outlet index for the source, and the
node index and inlet index for the destination. Throughout the interpreter, we
will often use the names $(src, outl)$ and $(dst, inl)$ to refer to those
indices.

\begin{hscode}\SaveRestoreHook
\column{B}{@{}>{\hspre}l<{\hspost}@{}}%
\column{4}{@{}>{\hspre}l<{\hspost}@{}}%
\column{E}{@{}>{\hspre}l<{\hspost}@{}}%
\>[B]{}\mathbf{data}\;(\cdot \rhd \cdot )\mathrel{=}((\Conid{Int},\Conid{Int})\rhd (\Conid{Int},\Conid{Int})){}\<[E]%
\\
\>[B]{}\hsindent{4}{}\<[4]%
\>[4]{}\mathbf{deriving}\;\Conid{Show}{}\<[E]%
\ColumnHook
\end{hscode}\resethooks

\subsection{Representation of states}

The representation of a state in our interpreter is a structure containing the
following values: the step count, which will double as our timestamp, since
Pure Data has time-based execution; the state for each node; the text contents
of the Pd logger window; and future events scheduled by the interpreter.

\begin{hscode}\SaveRestoreHook
\column{B}{@{}>{\hspre}l<{\hspost}@{}}%
\column{4}{@{}>{\hspre}l<{\hspost}@{}}%
\column{17}{@{}>{\hspre}l<{\hspost}@{}}%
\column{20}{@{}>{\hspre}l<{\hspost}@{}}%
\column{30}{@{}>{\hspre}l<{\hspost}@{}}%
\column{E}{@{}>{\hspre}l<{\hspost}@{}}%
\>[B]{}\mathbf{data}\;\Conid{PdState}\mathrel{=}{}\<[17]%
\>[17]{}\Conid{PdState}\;\{\mskip1.5mu {}\<[E]%
\\
\>[17]{}\hsindent{3}{}\<[20]%
\>[20]{}\Varid{sTs}{}\<[30]%
\>[30]{}\mathbin{::}\Conid{Int},{}\<[E]%
\\
\>[17]{}\hsindent{3}{}\<[20]%
\>[20]{}\Varid{sNStates}{}\<[30]%
\>[30]{}\mathbin{::}(\Conid{Seq}\;\Conid{PdNodeState}),{}\<[E]%
\\
\>[17]{}\hsindent{3}{}\<[20]%
\>[20]{}\Varid{sLog}{}\<[30]%
\>[30]{}\mathbin{::}[\mskip1.5mu \Conid{String}\mskip1.5mu],{}\<[E]%
\\
\>[17]{}\hsindent{3}{}\<[20]%
\>[20]{}\Varid{sSched}{}\<[30]%
\>[30]{}\mathbin{::}[\mskip1.5mu \Conid{PdEvent}\mskip1.5mu]{}\<[E]%
\\
\>[17]{}\mskip1.5mu\}{}\<[E]%
\\
\>[B]{}\hsindent{4}{}\<[4]%
\>[4]{}\mathbf{deriving}\;\Conid{Show}{}\<[E]%
\ColumnHook
\end{hscode}\resethooks

The state for each node, on its turn, contains a sequence of atom buffers,
one for each inlet, and an internal memory (represented as a list of atoms).
Memory consumption during execution is therefore variable, characterizing
a dynamic dataflow model.

\begin{hscode}\SaveRestoreHook
\column{B}{@{}>{\hspre}l<{\hspost}@{}}%
\column{4}{@{}>{\hspre}l<{\hspost}@{}}%
\column{E}{@{}>{\hspre}l<{\hspost}@{}}%
\>[B]{}\mathbf{data}\;\Conid{PdNodeState}\mathrel{=}\Conid{PdNodeState}\;(\Conid{Seq}\;[\mskip1.5mu \Conid{PdAtom}\mskip1.5mu])\;[\mskip1.5mu \Conid{PdAtom}\mskip1.5mu]{}\<[E]%
\\
\>[B]{}\hsindent{4}{}\<[4]%
\>[4]{}\mathbf{deriving}\;\Conid{Show}{}\<[E]%
\ColumnHook
\end{hscode}\resethooks

We represent events with a timestamp, the node index indicating which node was
triggered, and a list of atoms representing the event data (such as the number
entered by the user in an atom box).

\begin{hscode}\SaveRestoreHook
\column{B}{@{}>{\hspre}l<{\hspost}@{}}%
\column{4}{@{}>{\hspre}l<{\hspost}@{}}%
\column{17}{@{}>{\hspre}l<{\hspost}@{}}%
\column{20}{@{}>{\hspre}l<{\hspost}@{}}%
\column{27}{@{}>{\hspre}l<{\hspost}@{}}%
\column{E}{@{}>{\hspre}l<{\hspost}@{}}%
\>[B]{}\mathbf{data}\;\Conid{PdEvent}\mathrel{=}{}\<[17]%
\>[17]{}\Conid{PdEvent}\;\{\mskip1.5mu {}\<[E]%
\\
\>[17]{}\hsindent{3}{}\<[20]%
\>[20]{}\Varid{eTs}{}\<[27]%
\>[27]{}\mathbin{::}\Conid{Int},{}\<[E]%
\\
\>[17]{}\hsindent{3}{}\<[20]%
\>[20]{}\Varid{eNidx}{}\<[27]%
\>[27]{}\mathbin{::}\Conid{Int},{}\<[E]%
\\
\>[17]{}\hsindent{3}{}\<[20]%
\>[20]{}\Varid{eArg}{}\<[27]%
\>[27]{}\mathbin{::}[\mskip1.5mu \Conid{PdAtom}\mskip1.5mu]{}\<[E]%
\\
\>[17]{}\mskip1.5mu\}{}\<[E]%
\\
\>[B]{}\hsindent{4}{}\<[4]%
\>[4]{}\mathbf{deriving}\;(\Conid{Show},\Conid{Eq},\Conid{Ord}){}\<[E]%
\ColumnHook
\end{hscode}\resethooks

\subsection{Execution}

The execution mode of Pure Data is data-driven. The user triggers events via
its interface, and those events cause a cascading series of firings. The user
may trigger events by clicking nodes, entering numbers (or using MIDI devices,
which are equivalent to entering numbers).

\subsubsection{Main loop}

The execution of the interpreter, therefore, is a loop of evaluation steps.
The driver function takes a number of steps, the patch to run, a list of timed
events, accumulating a list of states. We are interested in all states, not
only the final one, because we want to be able to inspect the results of the
execution over time.

Note that the patch itself, \ensuremath{\Varid{p}}, remains unchanged over time. This
is typical of a language with liveness level 2: the patch cannot be modified
during execution.

\begin{hscode}\SaveRestoreHook
\column{B}{@{}>{\hspre}l<{\hspost}@{}}%
\column{4}{@{}>{\hspre}l<{\hspost}@{}}%
\column{7}{@{}>{\hspre}l<{\hspost}@{}}%
\column{10}{@{}>{\hspre}l<{\hspost}@{}}%
\column{13}{@{}>{\hspre}l<{\hspost}@{}}%
\column{E}{@{}>{\hspre}l<{\hspost}@{}}%
\>[B]{}\Varid{runSteps}\mathbin{::}\Conid{Int}\to \Conid{PdPatch}\to [\mskip1.5mu \Conid{PdEvent}\mskip1.5mu]\to [\mskip1.5mu \Conid{PdState}\mskip1.5mu]{}\<[E]%
\\
\>[B]{}\Varid{runSteps}\;\Varid{nSteps}\;\Varid{p}\;\Varid{events}\mathrel{=}{}\<[E]%
\\
\>[B]{}\hsindent{4}{}\<[4]%
\>[4]{}\Varid{reverse}\mathbin{\$}\Varid{snd}\mathbin{\$}\Varid{foldl'}\;\Varid{acc}\;(\Varid{events},[\mskip1.5mu \Varid{initialState}\;\Varid{p}\mskip1.5mu])\;[\mskip1.5mu \mathrm{0}\mathinner{\ldotp\ldotp}(\Varid{nSteps}\mathbin{-}\mathrm{1})\mskip1.5mu]{}\<[E]%
\\
\>[B]{}\hsindent{4}{}\<[4]%
\>[4]{}\mathbf{where}{}\<[E]%
\\
\>[4]{}\hsindent{3}{}\<[7]%
\>[7]{}\Varid{absTime}\mathbin{::}[\mskip1.5mu \Conid{PdEvent}\mskip1.5mu]\to \Conid{Int}\to [\mskip1.5mu \Conid{PdEvent}\mskip1.5mu]{}\<[E]%
\\
\>[4]{}\hsindent{3}{}\<[7]%
\>[7]{}\Varid{absTime}\;\Varid{evs}\;\Varid{ts}\mathrel{=}\Varid{map}\;(\lambda \Varid{e}\to \Varid{e}\;\{\mskip1.5mu \Varid{eTs}\mathrel{=}(\Varid{eTs}\;\Varid{e})\mathbin{+}\Varid{ts}\mskip1.5mu\})\;\Varid{evs}{}\<[E]%
\\
\>[4]{}\hsindent{3}{}\<[7]%
\>[7]{}\Varid{acc}\mathbin{::}([\mskip1.5mu \Conid{PdEvent}\mskip1.5mu],[\mskip1.5mu \Conid{PdState}\mskip1.5mu])\to \Conid{Int}\to ([\mskip1.5mu \Conid{PdEvent}\mskip1.5mu],[\mskip1.5mu \Conid{PdState}\mskip1.5mu]){}\<[E]%
\\
\>[4]{}\hsindent{3}{}\<[7]%
\>[7]{}\Varid{acc}\;(\Varid{events},\Varid{states}\mathord{@}(\Varid{s}\mathbin{:}\Varid{ss}))\;\Varid{step}\mathrel{=}{}\<[E]%
\\
\>[7]{}\hsindent{3}{}\<[10]%
\>[10]{}(\Varid{sort}\;(events_{next}\plus \Varid{absTime}\;(\Varid{sSched}\;\Varid{s'})\;\Varid{step}),\Varid{s'}\mathbin{:}\Varid{states}){}\<[E]%
\\
\>[7]{}\hsindent{3}{}\<[10]%
\>[10]{}\mathbf{where}{}\<[E]%
\\
\>[10]{}\hsindent{3}{}\<[13]%
\>[13]{}(events_{curr},events_{next})\mathrel{=}\Varid{span}\;(\lambda (\Conid{PdEvent}\;\Varid{ts}\;\anonymous \;\anonymous )\to \Varid{ts}\equiv \Varid{step})\;\Varid{events}{}\<[E]%
\\
\>[10]{}\hsindent{3}{}\<[13]%
\>[13]{}\Varid{s'}\mathrel{=}\Varid{runStep}\;\Varid{p}\;(\Varid{s}\;\{\mskip1.5mu \Varid{sSched}\mathrel{=}[\mskip1.5mu \mskip1.5mu]\mskip1.5mu\})\;events_{curr}{}\<[E]%
\ColumnHook
\end{hscode}\resethooks

The loop above extracts the sublist of relevant events for the current
timestamp, and hands it over to the main evaluation function, \ensuremath{\Varid{runStep}},
which, given a patch, the current state, and a list of events, produces
a new state.

Processing a step may produce new future events to be scheduled. These are
sorted along with the existing events of the input. Runtime events are
produced by the interpreter using relative timestamps (where 0 means ``now''),
so we adjust them to absolute time using auxiliary function \ensuremath{\Varid{adjTime}}.

The function \ensuremath{\Varid{runStep}} processes events and the DSP tree. Following the
specified semantics of Pure Data, this happens in an alternating fashion: all
pending messages for a given timestamp are handled, and then the entire DSP
tree is processed.

\begin{hscode}\SaveRestoreHook
\column{B}{@{}>{\hspre}l<{\hspost}@{}}%
\column{4}{@{}>{\hspre}l<{\hspost}@{}}%
\column{7}{@{}>{\hspre}l<{\hspost}@{}}%
\column{13}{@{}>{\hspre}l<{\hspost}@{}}%
\column{E}{@{}>{\hspre}l<{\hspost}@{}}%
\>[B]{}\Varid{runStep}\mathbin{::}\Conid{PdPatch}\to \Conid{PdState}\to [\mskip1.5mu \Conid{PdEvent}\mskip1.5mu]\to \Conid{PdState}{}\<[E]%
\\
\>[B]{}\Varid{runStep}\;\Varid{p}\;\Varid{s}\;\Varid{events}\mathrel{=}{}\<[E]%
\\
\>[B]{}\hsindent{4}{}\<[4]%
\>[4]{}\mathbf{let}{}\<[E]%
\\
\>[4]{}\hsindent{3}{}\<[7]%
\>[7]{}\Varid{s'}\mathrel{=}\Varid{runImmediateEvents}\;\Varid{p}\mathbin{\$}\Varid{foldl'}\;(\Varid{runEvent}\;\Varid{p})\;\Varid{s}\;\Varid{events}{}\<[E]%
\\
\>[4]{}\hsindent{3}{}\<[7]%
\>[7]{}\Varid{s''}\mathrel{=}\mathbf{if}\;(\Varid{sTs}\;\Varid{s})\mathbin{\Varid{`mod`}}\mathrm{2}\equiv \mathrm{0}{}\<[E]%
\\
\>[7]{}\hsindent{6}{}\<[13]%
\>[13]{}\mathbf{then}\;\Varid{runDspTree}\;\Varid{p}\;\Varid{s'}{}\<[E]%
\\
\>[7]{}\hsindent{6}{}\<[13]%
\>[13]{}\mathbf{else}\;\Varid{s'}{}\<[E]%
\\
\>[B]{}\hsindent{4}{}\<[4]%
\>[4]{}\mathbf{in}{}\<[E]%
\\
\>[4]{}\hsindent{3}{}\<[7]%
\>[7]{}\Varid{s''}\;\{\mskip1.5mu \Varid{sTs}\mathrel{=}(\Varid{sTs}\;\Varid{s})\mathbin{+}\mathrm{1}\mskip1.5mu\}{}\<[E]%
\ColumnHook
\end{hscode}\resethooks

In our model, the DSP tree is processed at half the rate of the message-based
events (hence, \ensuremath{\Varid{runDspTree}} is called at every other run of
\ensuremath{\Varid{runStep}}). Assuming that a step in our interpreter is 1~ms, this means
the DSP engine runs once every 2~ms (the default configuration of Pd runs the
engine every 1.45~ms; with a 64-sample buffer, this amounts to an audio sample
rate of 44,100~Hz \textemdash{} with this simplification in our interpreter, we get
36,000~Hz).

The Pure Data documentation specifies that "In the middle of a message cascade
you may schedule another one at a delay of zero. This delayed cascade happens
after the present cascade has finished, but at the same logical time". So,
events scheduled during the current step with a relative timestamp set to zero
are immediately executed before running the DSP tree:

\begin{hscode}\SaveRestoreHook
\column{B}{@{}>{\hspre}l<{\hspost}@{}}%
\column{4}{@{}>{\hspre}l<{\hspost}@{}}%
\column{7}{@{}>{\hspre}l<{\hspost}@{}}%
\column{E}{@{}>{\hspre}l<{\hspost}@{}}%
\>[B]{}\Varid{runImmediateEvents}\mathbin{::}\Conid{PdPatch}\to \Conid{PdState}\to \Conid{PdState}{}\<[E]%
\\
\>[B]{}\Varid{runImmediateEvents}\;\Varid{p}\;\Varid{s}\mathrel{=}{}\<[E]%
\\
\>[B]{}\hsindent{4}{}\<[4]%
\>[4]{}\mathbf{let}\;\Varid{z}\mathrel{=}[\mskip1.5mu \Varid{ev}\mid \Varid{ev}\leftarrow (\Varid{sSched}\;\Varid{s}),\Varid{eTs}\;\Varid{ev}\equiv \mathrm{0}\mskip1.5mu]{}\<[E]%
\\
\>[B]{}\hsindent{4}{}\<[4]%
\>[4]{}\mathbf{in}\;\mathbf{if}\;\Varid{z}\equiv [\mskip1.5mu \mskip1.5mu]{}\<[E]%
\\
\>[4]{}\hsindent{3}{}\<[7]%
\>[7]{}\mathbf{then}\;\Varid{s}{}\<[E]%
\\
\>[4]{}\hsindent{3}{}\<[7]%
\>[7]{}\mathbf{else}\;\Varid{runStep}\;\Varid{p}\;\Varid{s}\;\Varid{z}{}\<[E]%
\ColumnHook
\end{hscode}\resethooks

\subsubsection{Event processing}

Two kinds of events can be triggered by the user. Message boxes may be
clicked, processing all commands stored inside them, or new numeric values
may be entered into atom boxes. We do it producing a synthetic firing
of the relevant node.

\begin{hscode}\SaveRestoreHook
\column{B}{@{}>{\hspre}l<{\hspost}@{}}%
\column{4}{@{}>{\hspre}l<{\hspost}@{}}%
\column{E}{@{}>{\hspre}l<{\hspost}@{}}%
\>[B]{}\Varid{runEvent}\mathbin{::}\Conid{PdPatch}\to \Conid{PdState}\to \Conid{PdEvent}\to \Conid{PdState}{}\<[E]%
\\
\>[B]{}\Varid{runEvent}\;\Varid{p}\;\Varid{s}\;\Varid{event}\mathord{@}(\Conid{PdEvent}\;\Varid{ts}\;i_{N}\;\Varid{args})\mathrel{=}{}\<[E]%
\\
\>[B]{}\hsindent{4}{}\<[4]%
\>[4]{}\Varid{fire}\;\Varid{p}\;(\Varid{index}\;(\Varid{pNodes}\;\Varid{p})\;i_{N})\;\Varid{args}\;(i_{N},\mathrm{0})\;\Varid{s}{}\<[E]%
\ColumnHook
\end{hscode}\resethooks

The \ensuremath{\Varid{fire}} function invokes the appropriate action for a node, producing
a new state.

\begin{hscode}\SaveRestoreHook
\column{B}{@{}>{\hspre}l<{\hspost}@{}}%
\column{E}{@{}>{\hspre}l<{\hspost}@{}}%
\>[B]{}\Varid{fire}\mathbin{::}\Conid{PdPatch}\to \Conid{PdNode}\to [\mskip1.5mu \Conid{PdAtom}\mskip1.5mu]\to (\Conid{Int},\Conid{Int})\to \Conid{PdState}\to \Conid{PdState}{}\<[E]%
\ColumnHook
\end{hscode}\resethooks

Depending on the type of node, we perform different actions. For message
boxes, we feed the incoming atoms into the inlet, and then we fold over its
triggering its commands, like when they are clicked by the user. As we will
see below in the definition of \ensuremath{\Varid{runCommand}}, this may fire further nodes
either directly or indirectly.

\begin{hscode}\SaveRestoreHook
\column{B}{@{}>{\hspre}l<{\hspost}@{}}%
\column{4}{@{}>{\hspre}l<{\hspost}@{}}%
\column{7}{@{}>{\hspre}l<{\hspost}@{}}%
\column{E}{@{}>{\hspre}l<{\hspost}@{}}%
\>[B]{}\Varid{fire}\;\Varid{p}\;(\Conid{PdMsgBox}\;\Varid{cmds})\;\Varid{atoms}\;(i_{N},\Varid{inl})\;\Varid{s}\mathrel{=}{}\<[E]%
\\
\>[B]{}\hsindent{4}{}\<[4]%
\>[4]{}\mathbf{let}{}\<[E]%
\\
\>[4]{}\hsindent{3}{}\<[7]%
\>[7]{}(\Conid{PdNodeState}\;\Varid{ins}\;\Varid{mem})\mathrel{=}\Varid{index}\;(\Varid{sNStates}\;\Varid{s})\;i_{N}{}\<[E]%
\\
\>[4]{}\hsindent{3}{}\<[7]%
\>[7]{}\Varid{ns'}\mathrel{=}\Conid{PdNodeState}\;(\Varid{update}\;\Varid{inl}\;\Varid{atoms}\;\Varid{ins})\;\Varid{mem}{}\<[E]%
\\
\>[4]{}\hsindent{3}{}\<[7]%
\>[7]{}\Varid{s'}\mathrel{=}\Varid{s}\;\{\mskip1.5mu \Varid{sNStates}\mathrel{=}(\Varid{update}\;i_{N}\;\Varid{ns'}\;(\Varid{sNStates}\;\Varid{s}))\mskip1.5mu\}{}\<[E]%
\\
\>[B]{}\hsindent{4}{}\<[4]%
\>[4]{}\mathbf{in}{}\<[E]%
\\
\>[4]{}\hsindent{3}{}\<[7]%
\>[7]{}\Varid{foldl'}\;(\Varid{runCommand}\;\Varid{p}\;i_{N})\;\Varid{s'}\;\Varid{cmds}{}\<[E]%
\ColumnHook
\end{hscode}\resethooks

For objects and atom boxes, we hand over the incoming data to the
\ensuremath{\Varid{sendMsg}} handler function, which implements the various behaviors
supported by different Pure Data objects. The function \ensuremath{\Varid{sendMsg}}
returns a tuple with the updated node state, log outputs produced (if any),
data to be sent via outlets and new events to be scheduled. We update the
state with this data, adjusting the node index of the returned events to point
them to that of the current node (\ensuremath{i_{N}}): a node can only schedule events for
itself. Finally, we propagate the data through the outlets, processing them
from right to left, as mandated by the Pure Data specification.

\begin{hscode}\SaveRestoreHook
\column{B}{@{}>{\hspre}l<{\hspost}@{}}%
\column{4}{@{}>{\hspre}l<{\hspost}@{}}%
\column{7}{@{}>{\hspre}l<{\hspost}@{}}%
\column{10}{@{}>{\hspre}l<{\hspost}@{}}%
\column{20}{@{}>{\hspre}l<{\hspost}@{}}%
\column{34}{@{}>{\hspre}l<{\hspost}@{}}%
\column{E}{@{}>{\hspre}l<{\hspost}@{}}%
\>[B]{}\Varid{fire}\;\Varid{p}\;\Varid{node}\;\Varid{atoms}\;(i_{N},\Varid{inl})\;\Varid{s}\mathrel{=}{}\<[E]%
\\
\>[B]{}\hsindent{4}{}\<[4]%
\>[4]{}\mathbf{let}{}\<[E]%
\\
\>[4]{}\hsindent{3}{}\<[7]%
\>[7]{}\Varid{ns}\mathrel{=}\Varid{index}\;(\Varid{sNStates}\;\Varid{s})\;i_{N}{}\<[E]%
\\
\>[4]{}\hsindent{3}{}\<[7]%
\>[7]{}(\Varid{ns'},\Varid{logw'},\Varid{outlets},\Varid{evs})\mathrel{=}\Varid{sendMsg}\;\Varid{node}\;\Varid{atoms}\;\Varid{inl}\;\Varid{ns}{}\<[E]%
\\
\>[4]{}\hsindent{3}{}\<[7]%
\>[7]{}\Varid{s'}\mathrel{=}\Varid{s}\;\{\mskip1.5mu {}\<[E]%
\\
\>[7]{}\hsindent{3}{}\<[10]%
\>[10]{}\Varid{sNStates}{}\<[20]%
\>[20]{}\mathrel{=}\Varid{update}\;i_{N}\;\Varid{ns'}\;(\Varid{sNStates}\;\Varid{s}),{}\<[E]%
\\
\>[7]{}\hsindent{3}{}\<[10]%
\>[10]{}\Varid{sLog}{}\<[20]%
\>[20]{}\mathrel{=}(\Varid{sLog}\;\Varid{s}){}\<[34]%
\>[34]{}\plus \Varid{logw'},{}\<[E]%
\\
\>[7]{}\hsindent{3}{}\<[10]%
\>[10]{}\Varid{sSched}{}\<[20]%
\>[20]{}\mathrel{=}(\Varid{sSched}\;\Varid{s}){}\<[34]%
\>[34]{}\plus (\Varid{map}\;(\lambda \Varid{e}\to \Varid{e}\;\{\mskip1.5mu \Varid{eNidx}\mathrel{=}i_{N}\mskip1.5mu\})\;\Varid{evs}){}\<[E]%
\\
\>[4]{}\hsindent{3}{}\<[7]%
\>[7]{}\mskip1.5mu\}{}\<[E]%
\\[\blanklineskip]%
\>[4]{}\hsindent{3}{}\<[7]%
\>[7]{}\Varid{propagate}\mathbin{::}\Conid{PdState}\to ([\mskip1.5mu \Conid{PdAtom}\mskip1.5mu],\Conid{Int})\to \Conid{PdState}{}\<[E]%
\\
\>[4]{}\hsindent{3}{}\<[7]%
\>[7]{}\Varid{propagate}\;\Varid{s}\;(\Varid{atoms},\Varid{outl})\mathrel{=}{}\<[E]%
\\
\>[7]{}\hsindent{3}{}\<[10]%
\>[10]{}\mathbf{if}\;\Varid{atoms}\equiv [\mskip1.5mu \mskip1.5mu]{}\<[E]%
\\
\>[7]{}\hsindent{3}{}\<[10]%
\>[10]{}\mathbf{then}\;\Varid{s}{}\<[E]%
\\
\>[7]{}\hsindent{3}{}\<[10]%
\>[10]{}\mathbf{else}\;\Varid{forEachInOutlet}\;\Varid{p}\;(i_{N},\Varid{outl})\;\Varid{atoms}\;\Varid{s}{}\<[E]%
\\
\>[B]{}\hsindent{4}{}\<[4]%
\>[4]{}\mathbf{in}{}\<[E]%
\\
\>[4]{}\hsindent{3}{}\<[7]%
\>[7]{}\Varid{foldl'}\;\Varid{propagate}\;\Varid{s'}\;(\Varid{zip}\;(\Varid{reverse}\;\Varid{outlets})\;[\mskip1.5mu \Varid{length}\;\Varid{outlets}\mathbin{-}\mathrm{1}\mathinner{\ldotp\ldotp}\mathrm{0}\mskip1.5mu]){}\<[E]%
\ColumnHook
\end{hscode}\resethooks

When propagating data, we send it to every connected outlet of a node. A node
may have multiple outlets and multiple nodes can be connected to a single
outlet. This function takes the patch, a $\left(node, outlet\right)$ pair of
indices indicating the source of the data, the data itself (a list of atoms),
and the current state. It folds over the list of connections of the patch,
firing the data to the appropriate inlets of all matching connections.

\begin{hscode}\SaveRestoreHook
\column{B}{@{}>{\hspre}l<{\hspost}@{}}%
\column{4}{@{}>{\hspre}l<{\hspost}@{}}%
\column{7}{@{}>{\hspre}l<{\hspost}@{}}%
\column{10}{@{}>{\hspre}l<{\hspost}@{}}%
\column{29}{@{}>{\hspre}l<{\hspost}@{}}%
\column{E}{@{}>{\hspre}l<{\hspost}@{}}%
\>[B]{}\Varid{forEachInOutlet}\mathbin{::}\Conid{PdPatch}\to (\Conid{Int},\Conid{Int})\to [\mskip1.5mu \Conid{PdAtom}\mskip1.5mu]\to \Conid{PdState}\to \Conid{PdState}{}\<[E]%
\\
\>[B]{}\Varid{forEachInOutlet}\;\Varid{p}\;\Varid{srcPair}\;\Varid{atoms}\;\Varid{s}\mathrel{=}{}\<[E]%
\\
\>[B]{}\hsindent{4}{}\<[4]%
\>[4]{}\Varid{foldl'}\;\Varid{handle}\;\Varid{s}\;(\Varid{pConns}\;\Varid{p}){}\<[E]%
\\
\>[B]{}\hsindent{4}{}\<[4]%
\>[4]{}\mathbf{where}{}\<[E]%
\\
\>[4]{}\hsindent{3}{}\<[7]%
\>[7]{}\Varid{handle}\mathbin{::}\Conid{PdState}\to (\cdot \rhd \cdot )\to \Conid{PdState}{}\<[E]%
\\
\>[4]{}\hsindent{3}{}\<[7]%
\>[7]{}\Varid{handle}\;\Varid{s}\;(\Varid{from}\rhd (\Varid{to}\mathord{@}(\Varid{dst},\Varid{inl}))){}\<[E]%
\\
\>[7]{}\hsindent{3}{}\<[10]%
\>[10]{}\mid \Varid{srcPair}\equiv \Varid{from}{}\<[29]%
\>[29]{}\mathrel{=}\Varid{fire}\;\Varid{p}\;(\Varid{index}\;(\Varid{pNodes}\;\Varid{p})\;\Varid{dst})\;\Varid{atoms}\;\Varid{to}\;\Varid{s}{}\<[E]%
\\
\>[7]{}\hsindent{3}{}\<[10]%
\>[10]{}\mid \Varid{otherwise}{}\<[29]%
\>[29]{}\mathrel{=}\Varid{s}{}\<[E]%
\ColumnHook
\end{hscode}\resethooks

Pure Data commands are written in its textual language. Commands may include
references to data obtained via inlets of the node using the $\$n$ notation.
For example, sending \texttt{10 20} to a message box containing \texttt{pitch
\$2 velocity \$1} connected to an object box \texttt{print} will print to the
log window the string \texttt{pitch 20 velocity 10}.

In function \ensuremath{\Varid{runCommand}} below, we run a given command \ensuremath{\Varid{cmd}} on a
node (with index \ensuremath{i_{N}}) by first obtaining the inlet data currently
stored in the node state. Then we perform $\$$-expansion on the command's
tokens. Then, based on the receiver of the message, we route it through the
graph (forwarding it to every outlet, in a classic dataflow fashion) or
symbolically, sending it to all objects configured as a receivers for the
given name.

\begin{hscode}\SaveRestoreHook
\column{B}{@{}>{\hspre}l<{\hspost}@{}}%
\column{4}{@{}>{\hspre}l<{\hspost}@{}}%
\column{7}{@{}>{\hspre}l<{\hspost}@{}}%
\column{10}{@{}>{\hspre}l<{\hspost}@{}}%
\column{E}{@{}>{\hspre}l<{\hspost}@{}}%
\>[B]{}\Varid{runCommand}\mathbin{::}\Conid{PdPatch}\to \Conid{Int}\to \Conid{PdState}\to \Conid{PdCmd}\to \Conid{PdState}{}\<[E]%
\\
\>[B]{}\Varid{runCommand}\;\Varid{p}\;i_{N}\;(\Conid{PdState}\;\Varid{ts}\;\Varid{nss}\;\Varid{logw}\;\Varid{evs})\;\Varid{cmd}\mathrel{=}{}\<[E]%
\\
\>[B]{}\hsindent{4}{}\<[4]%
\>[4]{}\mathbf{let}{}\<[E]%
\\
\>[4]{}\hsindent{3}{}\<[7]%
\>[7]{}(\Conid{PdNodeState}\;\Varid{ins}\;\Varid{mem})\mathrel{=}\Varid{index}\;\Varid{nss}\;i_{N}{}\<[E]%
\\
\>[4]{}\hsindent{3}{}\<[7]%
\>[7]{}\Varid{inletData}\mathrel{=}\Varid{index}\;\Varid{ins}\;\mathrm{0}{}\<[E]%
\\
\>[4]{}\hsindent{3}{}\<[7]%
\>[7]{}(\Varid{recv},\Varid{atoms})\mathrel{=}\Varid{dollarExpansion}\;\Varid{cmd}\;\Varid{inletData}{}\<[E]%
\\
\>[4]{}\hsindent{3}{}\<[7]%
\>[7]{}\Varid{ns'}\mathrel{=}\Conid{PdNodeState}\;(\Varid{update}\;\mathrm{0}\;[\mskip1.5mu \mskip1.5mu]\;\Varid{ins})\;\Varid{mem}{}\<[E]%
\\
\>[4]{}\hsindent{3}{}\<[7]%
\>[7]{}\Varid{nss'}\mathrel{=}\Varid{update}\;i_{N}\;\Varid{ns'}\;\Varid{nss}{}\<[E]%
\\
\>[4]{}\hsindent{3}{}\<[7]%
\>[7]{}\Varid{s'}\mathrel{=}\Conid{PdState}\;\Varid{ts}\;\Varid{nss'}\;\Varid{logw}\;\Varid{evs}{}\<[E]%
\\
\>[B]{}\hsindent{4}{}\<[4]%
\>[4]{}\mathbf{in}{}\<[E]%
\\
\>[4]{}\hsindent{3}{}\<[7]%
\>[7]{}\mathbf{case}\;\Varid{recv}\;\mathbf{of}{}\<[E]%
\\
\>[4]{}\hsindent{3}{}\<[7]%
\>[7]{}\Conid{PdToOutlet}\to {}\<[E]%
\\
\>[7]{}\hsindent{3}{}\<[10]%
\>[10]{}\Varid{forEachInOutlet}\;\Varid{p}\;(i_{N},\mathrm{0})\;\Varid{atoms}\;\Varid{s'}{}\<[E]%
\\
\>[4]{}\hsindent{3}{}\<[7]%
\>[7]{}\Conid{PdReceiver}\;\Varid{r}\to {}\<[E]%
\\
\>[7]{}\hsindent{3}{}\<[10]%
\>[10]{}\Varid{forEachReceiver}\;\Varid{p}\;\Varid{r}\;\Varid{atoms}\;\Varid{s'}{}\<[E]%
\\
\>[4]{}\hsindent{3}{}\<[7]%
\>[7]{}\Conid{PdReceiverErr}\to {}\<[E]%
\\
\>[7]{}\hsindent{3}{}\<[10]%
\>[10]{}\Varid{printOut}\;[\mskip1.5mu \Conid{PdSymbol}\;\text{\tt \char34 \$1:~symbol~needed~as~message~destination\char34}\mskip1.5mu]\;\Varid{s'}{}\<[E]%
\ColumnHook
\end{hscode}\resethooks

The process of $\$$-expansion is a simple substitution, where the receiver
must be a string. Invalid indices are converted to zero. (In Pure Data, they
also produce an error message to the log window, but here we omit this for
brevity.) We also handle here a few syntactic shortcuts: a messages with a
sole number like \texttt{1.0} expands to \texttt{float 1.0}; lists starting
with a number get the prefix \texttt{list}.

\begin{hscode}\SaveRestoreHook
\column{B}{@{}>{\hspre}l<{\hspost}@{}}%
\column{4}{@{}>{\hspre}l<{\hspost}@{}}%
\column{7}{@{}>{\hspre}l<{\hspost}@{}}%
\column{10}{@{}>{\hspre}l<{\hspost}@{}}%
\column{13}{@{}>{\hspre}l<{\hspost}@{}}%
\column{16}{@{}>{\hspre}l<{\hspost}@{}}%
\column{27}{@{}>{\hspre}l<{\hspost}@{}}%
\column{28}{@{}>{\hspre}l<{\hspost}@{}}%
\column{41}{@{}>{\hspre}l<{\hspost}@{}}%
\column{E}{@{}>{\hspre}l<{\hspost}@{}}%
\>[B]{}\Varid{ffor}\;\Varid{a}\;\Varid{f}\mathrel{=}\Varid{fmap}\;\Varid{f}\;\Varid{a}{}\<[E]%
\\[\blanklineskip]%
\>[B]{}\Varid{dollarExpansion}\mathbin{::}\Conid{PdCmd}\to [\mskip1.5mu \Conid{PdAtom}\mskip1.5mu]\to (\Conid{PdReceiver},[\mskip1.5mu \Conid{PdAtom}\mskip1.5mu]){}\<[E]%
\\
\>[B]{}\Varid{dollarExpansion}\;(\Conid{PdCmd}\;\Varid{recv}\;\Varid{tokens})\;\Varid{inlData}\mathrel{=}{}\<[E]%
\\
\>[B]{}\hsindent{4}{}\<[4]%
\>[4]{}(\Varid{recv'},\Varid{atoms'}){}\<[E]%
\\
\>[B]{}\hsindent{4}{}\<[4]%
\>[4]{}\mathbf{where}{}\<[E]%
\\
\>[4]{}\hsindent{3}{}\<[7]%
\>[7]{}\Varid{inlAt}\;\Varid{n}\mathrel{=}\mathbf{if}\;\Varid{n}\mathbin{<}\Varid{length}\;\Varid{inlData}\;\mathbf{then}\;\Varid{inlData}\mathbin{!!}\Varid{n}\;\mathbf{else}\;\Conid{PdFloat}\;\mathrm{0}{}\<[E]%
\\
\>[4]{}\hsindent{3}{}\<[7]%
\>[7]{}\Varid{recv'}\mathrel{=}{}\<[E]%
\\
\>[7]{}\hsindent{3}{}\<[10]%
\>[10]{}\mathbf{case}\;\Varid{recv}\;\mathbf{of}{}\<[E]%
\\
\>[7]{}\hsindent{3}{}\<[10]%
\>[10]{}\Conid{PdRDollar}\;\Varid{n}\to {}\<[E]%
\\
\>[10]{}\hsindent{3}{}\<[13]%
\>[13]{}\mathbf{case}\;\Varid{inlAt}\;\Varid{n}\;\mathbf{of}{}\<[E]%
\\
\>[13]{}\hsindent{3}{}\<[16]%
\>[16]{}\Conid{PdSymbol}\;\Varid{s}{}\<[28]%
\>[28]{}\to \Conid{PdReceiver}\;\Varid{s}{}\<[E]%
\\
\>[13]{}\hsindent{3}{}\<[16]%
\>[16]{}\anonymous {}\<[28]%
\>[28]{}\to \Conid{PdReceiverErr}{}\<[E]%
\\
\>[7]{}\hsindent{3}{}\<[10]%
\>[10]{}\anonymous \to \Varid{recv}{}\<[E]%
\\
\>[4]{}\hsindent{3}{}\<[7]%
\>[7]{}\Varid{atoms'}\mathrel{=}{}\<[E]%
\\
\>[7]{}\hsindent{3}{}\<[10]%
\>[10]{}\Varid{normalize}\mathbin{\$}\Varid{ffor}\;\Varid{tokens}\;(\lambda \Varid{token}\to {}\<[E]%
\\
\>[10]{}\hsindent{3}{}\<[13]%
\>[13]{}\mathbf{case}\;\Varid{token}\;\mathbf{of}{}\<[E]%
\\
\>[10]{}\hsindent{3}{}\<[13]%
\>[13]{}\Conid{PdTDollar}\;\Varid{n}{}\<[27]%
\>[27]{}\to \Varid{inlAt}\;\Varid{n}{}\<[E]%
\\
\>[10]{}\hsindent{3}{}\<[13]%
\>[13]{}\Conid{PdTAtom}\;\Varid{atom}{}\<[27]%
\>[27]{}\to \Varid{atom}{}\<[E]%
\\
\>[7]{}\hsindent{3}{}\<[10]%
\>[10]{}){}\<[E]%
\\
\>[4]{}\hsindent{3}{}\<[7]%
\>[7]{}\Varid{normalize}\;\Varid{atoms}\mathord{@}[\mskip1.5mu \Conid{PdFloat}\;\Varid{f}\mskip1.5mu]{}\<[41]%
\>[41]{}\mathrel{=}(\Conid{PdSymbol}\;\text{\tt \char34 float\char34}\mathbin{:}\Varid{atoms}){}\<[E]%
\\
\>[4]{}\hsindent{3}{}\<[7]%
\>[7]{}\Varid{normalize}\;\Varid{atoms}\mathord{@}(\Conid{PdFloat}\;\Varid{f}\mathbin{:}\Varid{xs}){}\<[41]%
\>[41]{}\mathrel{=}(\Conid{PdSymbol}\;\text{\tt \char34 list\char34}\mathbin{:}\Varid{atoms}){}\<[E]%
\\
\>[4]{}\hsindent{3}{}\<[7]%
\>[7]{}\Varid{normalize}\;\Varid{atoms}{}\<[41]%
\>[41]{}\mathrel{=}\Varid{atoms}{}\<[E]%
\ColumnHook
\end{hscode}\resethooks

Indirect connections are handled similarly to outlet connections, but instead
of folding over the list of connections, we fold over the list of nodes,
looking for objects declared as \texttt{receive $name$}. Note that the search
happens over the statically-declared list of nodes of the patch. While it is
possible construct a message at runtime and determine the receiver
dynamically, it is not possible to change the identifier of a \texttt{receive}
node at runtime.

\begin{hscode}\SaveRestoreHook
\column{B}{@{}>{\hspre}l<{\hspost}@{}}%
\column{4}{@{}>{\hspre}l<{\hspost}@{}}%
\column{7}{@{}>{\hspre}l<{\hspost}@{}}%
\column{10}{@{}>{\hspre}l<{\hspost}@{}}%
\column{29}{@{}>{\hspre}l<{\hspost}@{}}%
\column{E}{@{}>{\hspre}l<{\hspost}@{}}%
\>[B]{}\Varid{forEachReceiver}\mathbin{::}\Conid{PdPatch}{}\<[29]%
\>[29]{}\to \Conid{String}{}\<[E]%
\\
\>[29]{}\to [\mskip1.5mu \Conid{PdAtom}\mskip1.5mu]{}\<[E]%
\\
\>[29]{}\to \Conid{PdState}\to \Conid{PdState}{}\<[E]%
\\
\>[B]{}\Varid{forEachReceiver}\;\Varid{p}\;\Varid{name}\;\Varid{atoms}\;\Varid{s}\mathrel{=}{}\<[E]%
\\
\>[B]{}\hsindent{4}{}\<[4]%
\>[4]{}\Varid{foldlWithIndex}\;\Varid{handle}\;\Varid{s}\;(\Varid{pNodes}\;\Varid{p}){}\<[E]%
\\
\>[B]{}\hsindent{4}{}\<[4]%
\>[4]{}\mathbf{where}{}\<[E]%
\\
\>[4]{}\hsindent{3}{}\<[7]%
\>[7]{}\Varid{handle}\mathbin{::}\Conid{PdState}\to \Conid{Int}\to \Conid{PdNode}\to \Conid{PdState}{}\<[E]%
\\
\>[4]{}\hsindent{3}{}\<[7]%
\>[7]{}\Varid{handle}\;\Varid{s}\;\Varid{dst}\;(\Conid{PdObj}\;(\Conid{PdSymbol}\;\text{\tt \char34 receive\char34}\mathbin{:}(\Conid{PdSymbol}\;\Varid{rname}\mathbin{:}\anonymous ))\;\anonymous \;\anonymous ){}\<[E]%
\\
\>[7]{}\hsindent{3}{}\<[10]%
\>[10]{}\mid \Varid{name}\equiv \Varid{rname}\mathrel{=}\Varid{forEachInOutlet}\;\Varid{p}\;(\Varid{dst},\mathrm{0})\;\Varid{atoms}\;\Varid{s}{}\<[E]%
\\
\>[4]{}\hsindent{3}{}\<[7]%
\>[7]{}\Varid{handle}\;\Varid{s}\;\anonymous \;\anonymous \mathrel{=}\Varid{s}{}\<[E]%
\ColumnHook
\end{hscode}\resethooks

\subsubsection{Audio processing}

The processing of audio nodes is very different from that of message nodes.
Before execution, the audio nodes are topologically sorted, producing an
order according to which they are evaluated on each DSP update. For simplicity,
we do not compute this order at the beginning of execution, and merely assume
it is given as an input (in the \ensuremath{\Varid{dspSort}} field of \ensuremath{\Varid{p}}).

As the list of nodes is traversed, each object is triggered (applying the
\ensuremath{\Varid{performDsp}} function) and then the new computed value of its audio
buffer is propagated to the inlets of the nodes to which they are connected. 

\begin{hscode}\SaveRestoreHook
\column{B}{@{}>{\hspre}l<{\hspost}@{}}%
\column{4}{@{}>{\hspre}l<{\hspost}@{}}%
\column{7}{@{}>{\hspre}l<{\hspost}@{}}%
\column{10}{@{}>{\hspre}l<{\hspost}@{}}%
\column{13}{@{}>{\hspre}l<{\hspost}@{}}%
\column{16}{@{}>{\hspre}l<{\hspost}@{}}%
\column{31}{@{}>{\hspre}l<{\hspost}@{}}%
\column{E}{@{}>{\hspre}l<{\hspost}@{}}%
\>[B]{}\Varid{runDspTree}\mathbin{::}\Conid{PdPatch}\to \Conid{PdState}\to \Conid{PdState}{}\<[E]%
\\
\>[B]{}\Varid{runDspTree}\;\Varid{p}\;\Varid{s}\mathrel{=}{}\<[E]%
\\
\>[B]{}\hsindent{4}{}\<[4]%
\>[4]{}\Varid{s}\;\{\mskip1.5mu \Varid{sNStates}\mathrel{=}\Varid{nss'}\mskip1.5mu\}{}\<[E]%
\\
\>[B]{}\hsindent{4}{}\<[4]%
\>[4]{}\mathbf{where}{}\<[E]%
\\
\>[4]{}\hsindent{3}{}\<[7]%
\>[7]{}\Varid{nss'}\mathrel{=}\Varid{foldl'}\;\Varid{handle}\;(\Varid{zeroDspInlets}\;(\Varid{sNStates}\;\Varid{s})\;(\Varid{pDspSort}\;\Varid{p}))\;(\Varid{pDspSort}\;\Varid{p}){}\<[E]%
\\[\blanklineskip]%
\>[4]{}\hsindent{3}{}\<[7]%
\>[7]{}\Varid{handle}\mathbin{::}(\Conid{Seq}\;\Conid{PdNodeState})\to \Conid{Int}\to (\Conid{Seq}\;\Conid{PdNodeState}){}\<[E]%
\\
\>[4]{}\hsindent{3}{}\<[7]%
\>[7]{}\Varid{handle}\;\Varid{nss}\;i_{N}\mathrel{=}{}\<[E]%
\\
\>[7]{}\hsindent{3}{}\<[10]%
\>[10]{}\Varid{foldl'}\;(\Varid{propagate}\;\Varid{outputs})\;\Varid{nss''}\;(\Varid{pConns}\;\Varid{p}){}\<[E]%
\\
\>[7]{}\hsindent{3}{}\<[10]%
\>[10]{}\mathbf{where}{}\<[E]%
\\
\>[10]{}\hsindent{3}{}\<[13]%
\>[13]{}\Varid{obj}\mathrel{=}\Varid{index}\;(\Varid{pNodes}\;\Varid{p})\;i_{N}{}\<[E]%
\\
\>[10]{}\hsindent{3}{}\<[13]%
\>[13]{}\Varid{ns}\mathord{@}(\Conid{PdNodeState}\;\Varid{ins}\;\Varid{mem})\mathrel{=}\Varid{index}\;\Varid{nss}\;i_{N}{}\<[E]%
\\
\>[10]{}\hsindent{3}{}\<[13]%
\>[13]{}(\Varid{outputs},\Varid{mem'})\mathrel{=}\Varid{performDsp}\;\Varid{obj}\;\Varid{ns}{}\<[E]%
\\
\>[10]{}\hsindent{3}{}\<[13]%
\>[13]{}\Varid{nss''}\mathrel{=}\Varid{update}\;i_{N}\;(\Conid{PdNodeState}\;\Varid{ins}\;\Varid{mem'})\;\Varid{nss}{}\<[E]%
\\
\>[10]{}\hsindent{3}{}\<[13]%
\>[13]{}\Varid{propagate}\mathbin{::}[\mskip1.5mu [\mskip1.5mu \Conid{PdAtom}\mskip1.5mu]\mskip1.5mu]\to (\Conid{Seq}\;\Conid{PdNodeState})\to (\cdot \rhd \cdot )\to (\Conid{Seq}\;\Conid{PdNodeState}){}\<[E]%
\\
\>[10]{}\hsindent{3}{}\<[13]%
\>[13]{}\Varid{propagate}\;\Varid{outputs}\;\Varid{nss}\;((\Varid{src},\Varid{outl})\rhd (\Varid{dst},\Varid{inl})){}\<[E]%
\\
\>[13]{}\hsindent{3}{}\<[16]%
\>[16]{}\mid \Varid{src}\equiv i_{N}{}\<[31]%
\>[31]{}\mathrel{=}\Varid{addToInlet}\;(\Varid{dst},\Varid{inl})\;(\Varid{outputs}\mathbin{!!}\Varid{outl})\;\Varid{nss}{}\<[E]%
\\
\>[13]{}\hsindent{3}{}\<[16]%
\>[16]{}\mid \Varid{otherwise}{}\<[31]%
\>[31]{}\mathrel{=}\Varid{nss}{}\<[E]%
\ColumnHook
\end{hscode}\resethooks

Each audio node has a 64-sample buffer that needs to be cleared before each
traversal. Note that this is different from handling inlets in message
objects: for message objects, the inlets become empty once consumed. Here, we
need the inlet buffers to be filled with zeros.

\begin{hscode}\SaveRestoreHook
\column{B}{@{}>{\hspre}l<{\hspost}@{}}%
\column{4}{@{}>{\hspre}l<{\hspost}@{}}%
\column{7}{@{}>{\hspre}l<{\hspost}@{}}%
\column{10}{@{}>{\hspre}l<{\hspost}@{}}%
\column{13}{@{}>{\hspre}l<{\hspost}@{}}%
\column{26}{@{}>{\hspre}l<{\hspost}@{}}%
\column{27}{@{}>{\hspre}l<{\hspost}@{}}%
\column{32}{@{}>{\hspre}l<{\hspost}@{}}%
\column{36}{@{}>{\hspre}l<{\hspost}@{}}%
\column{42}{@{}>{\hspre}l<{\hspost}@{}}%
\column{E}{@{}>{\hspre}l<{\hspost}@{}}%
\>[B]{}\Varid{zeroDspInlets}\mathbin{::}(\Conid{Seq}\;\Conid{PdNodeState})\to [\mskip1.5mu \Conid{Int}\mskip1.5mu]\to (\Conid{Seq}\;\Conid{PdNodeState}){}\<[E]%
\\
\>[B]{}\Varid{zeroDspInlets}\;\Varid{nss}\;\Varid{dspSort}\mathrel{=}{}\<[E]%
\\
\>[B]{}\hsindent{4}{}\<[4]%
\>[4]{}\Varid{fromList}\mathbin{\$}\Varid{clearNodes}\;\mathrm{0}\;(\Varid{toList}\;\Varid{nss})\;(\Varid{sort}\;\Varid{dspSort}){}\<[E]%
\\
\>[4]{}\hsindent{3}{}\<[7]%
\>[7]{}\mathbf{where}{}\<[E]%
\\
\>[7]{}\hsindent{3}{}\<[10]%
\>[10]{}\Varid{zeroInlets}\mathbin{::}\Conid{Int}\to (\Conid{Seq}\;[\mskip1.5mu \Conid{PdAtom}\mskip1.5mu]){}\<[E]%
\\
\>[7]{}\hsindent{3}{}\<[10]%
\>[10]{}\Varid{zeroInlets}\;\Varid{n}\mathrel{=}\Varid{fromList}\mathbin{\$}\Varid{replicate}\;\Varid{n}\;(\Varid{replicate}\;\mathrm{64}\;(\Conid{PdFloat}\;\mathrm{0.0})){}\<[E]%
\\[\blanklineskip]%
\>[7]{}\hsindent{3}{}\<[10]%
\>[10]{}\Varid{zeroState}\mathbin{::}\Conid{PdNodeState}\to \Conid{PdNodeState}{}\<[E]%
\\
\>[7]{}\hsindent{3}{}\<[10]%
\>[10]{}\Varid{zeroState}\;(\Conid{PdNodeState}\;\Varid{ins}\;\Varid{mem})\mathrel{=}{}\<[E]%
\\
\>[10]{}\hsindent{3}{}\<[13]%
\>[13]{}\Conid{PdNodeState}\;(\Varid{zeroInlets}\;(\Varid{\Conid{Seq}.length}\;\Varid{ins}))\;\Varid{mem}{}\<[E]%
\\[\blanklineskip]%
\>[7]{}\hsindent{3}{}\<[10]%
\>[10]{}\Varid{clearNodes}\mathbin{::}\Conid{Int}\to [\mskip1.5mu \Conid{PdNodeState}\mskip1.5mu]\to [\mskip1.5mu \Conid{Int}\mskip1.5mu]\to [\mskip1.5mu \Conid{PdNodeState}\mskip1.5mu]{}\<[E]%
\\
\>[7]{}\hsindent{3}{}\<[10]%
\>[10]{}\Varid{clearNodes}\;i_{N}\;(\Varid{st}\mathbin{:}\Varid{sts})\;\Varid{indices}\mathord{@}(\Varid{i}\mathbin{:}\Varid{is}){}\<[E]%
\\
\>[10]{}\hsindent{3}{}\<[13]%
\>[13]{}\mid i_{N}\equiv \Varid{i}{}\<[26]%
\>[26]{}\mathrel{=}\Varid{zeroState}\;\Varid{st}{}\<[42]%
\>[42]{}\mathbin{:}\Varid{clearNodes}\;(i_{N}\mathbin{+}\mathrm{1})\;\Varid{sts}\;\Varid{is}{}\<[E]%
\\
\>[10]{}\hsindent{3}{}\<[13]%
\>[13]{}\mid \Varid{otherwise}{}\<[26]%
\>[26]{}\mathrel{=}\Varid{st}{}\<[42]%
\>[42]{}\mathbin{:}\Varid{clearNodes}\;(i_{N}\mathbin{+}\mathrm{1})\;\Varid{sts}\;\Varid{indices}{}\<[E]%
\\
\>[7]{}\hsindent{3}{}\<[10]%
\>[10]{}\Varid{clearNodes}\;i_{N}\;{}\<[27]%
\>[27]{}\Varid{nss}\;{}\<[32]%
\>[32]{}[\mskip1.5mu \mskip1.5mu]{}\<[36]%
\>[36]{}\mathrel{=}\Varid{nss}{}\<[E]%
\\
\>[7]{}\hsindent{3}{}\<[10]%
\>[10]{}\Varid{clearNodes}\;i_{N}\;{}\<[27]%
\>[27]{}[\mskip1.5mu \mskip1.5mu]\;{}\<[32]%
\>[32]{}\anonymous {}\<[36]%
\>[36]{}\mathrel{=}[\mskip1.5mu \mskip1.5mu]{}\<[E]%
\ColumnHook
\end{hscode}\resethooks

The reason why we fill the inlets with zeros is because when multiple
nodes connect to the same inlet in a DSP object, additive synthesis
is performed: the values of the incoming buffer are added to the
current contents of the inlet buffer, subject to saturation (audio
values are internally floats between -1.0 and 1.0).

\begin{hscode}\SaveRestoreHook
\column{B}{@{}>{\hspre}l<{\hspost}@{}}%
\column{4}{@{}>{\hspre}l<{\hspost}@{}}%
\column{7}{@{}>{\hspre}l<{\hspost}@{}}%
\column{E}{@{}>{\hspre}l<{\hspost}@{}}%
\>[B]{}\Varid{addToInlet}\mathbin{::}(\Conid{Int},\Conid{Int})\to [\mskip1.5mu \Conid{PdAtom}\mskip1.5mu]\to (\Conid{Seq}\;\Conid{PdNodeState})\to (\Conid{Seq}\;\Conid{PdNodeState}){}\<[E]%
\\
\>[B]{}\Varid{addToInlet}\;(\Varid{dst},\Varid{inl})\;\Varid{atoms}\;\Varid{nss}\mathrel{=}\Varid{update}\;\Varid{dst}\;\Varid{ns'}\;\Varid{nss}{}\<[E]%
\\
\>[B]{}\hsindent{4}{}\<[4]%
\>[4]{}\mathbf{where}{}\<[E]%
\\
\>[4]{}\hsindent{3}{}\<[7]%
\>[7]{}\Varid{saturate}\;(\Conid{PdFloat}\;\Varid{f})\mathrel{=}\Conid{PdFloat}\;(\Varid{max}\;(\mathbin{-}\mathrm{1.0})\;(\Varid{min}\;\mathrm{1.0}\;\Varid{f})){}\<[E]%
\\[\blanklineskip]%
\>[4]{}\hsindent{3}{}\<[7]%
\>[7]{}\Varid{satSum}\;(\Conid{PdFloat}\;\Varid{a},\Conid{PdFloat}\;\Varid{b})\mathrel{=}\Varid{saturate}\mathbin{\$}\Conid{PdFloat}\;(\Varid{a}\mathbin{+}\Varid{b}){}\<[E]%
\\
\>[4]{}\hsindent{3}{}\<[7]%
\>[7]{}\Varid{ns}\mathord{@}(\Conid{PdNodeState}\;\Varid{ins}\;\Varid{mem})\mathrel{=}\Varid{index}\;\Varid{nss}\;\Varid{dst}{}\<[E]%
\\
\>[4]{}\hsindent{3}{}\<[7]%
\>[7]{}atoms_{old}\mathrel{=}\Varid{index}\;\Varid{ins}\;\Varid{inl}{}\<[E]%
\\
\>[4]{}\hsindent{3}{}\<[7]%
\>[7]{}atoms_{new}\mathrel{=}\Varid{fmap}\;\Varid{satSum}\;(\Varid{zip}\;atoms_{old}\;\Varid{atoms}){}\<[E]%
\\
\>[4]{}\hsindent{3}{}\<[7]%
\>[7]{}\Varid{ns'}\mathrel{=}\Conid{PdNodeState}\;(\Varid{update}\;\Varid{inl}\;atoms_{new}\;\Varid{ins})\;\Varid{mem}{}\<[E]%
\ColumnHook
\end{hscode}\resethooks

In Section~\ref{dsps} we will present \ensuremath{\Varid{performDsp}}, which implements the
various DSP objects supported by this interpreter.

\subsubsection{Initial state}

Finally, for completeness of the execution model, we present here the
functions that create the initial state.

\begin{hscode}\SaveRestoreHook
\column{B}{@{}>{\hspre}l<{\hspost}@{}}%
\column{4}{@{}>{\hspre}l<{\hspost}@{}}%
\column{7}{@{}>{\hspre}l<{\hspost}@{}}%
\column{10}{@{}>{\hspre}l<{\hspost}@{}}%
\column{21}{@{}>{\hspre}l<{\hspost}@{}}%
\column{30}{@{}>{\hspre}l<{\hspost}@{}}%
\column{64}{@{}>{\hspre}l<{\hspost}@{}}%
\column{E}{@{}>{\hspre}l<{\hspost}@{}}%
\>[B]{}\Varid{emptyInlets}\mathbin{::}\Conid{Int}\to \Conid{Seq}\;[\mskip1.5mu \Conid{PdAtom}\mskip1.5mu]{}\<[E]%
\\
\>[B]{}\Varid{emptyInlets}\;\Varid{n}\mathrel{=}\Varid{fromList}\;(\Varid{replicate}\;\Varid{n}\;[\mskip1.5mu \mskip1.5mu]){}\<[E]%
\\[\blanklineskip]%
\>[B]{}\Varid{initialState}\mathbin{::}\Conid{PdPatch}\to \Conid{PdState}{}\<[E]%
\\
\>[B]{}\Varid{initialState}\;(\Conid{PdPatch}\;\Varid{nodes}\;\anonymous \;\anonymous )\mathrel{=}\Conid{PdState}\;\mathrm{0}\;(\Varid{fmap}\;\Varid{emptyNode}\;\Varid{nodes})\;[\mskip1.5mu \mskip1.5mu]\;[\mskip1.5mu \mskip1.5mu]{}\<[E]%
\\
\>[B]{}\hsindent{4}{}\<[4]%
\>[4]{}\mathbf{where}{}\<[E]%
\\
\>[B]{}\hsindent{4}{}\<[4]%
\>[4]{}\Varid{emptyNode}\;\Varid{node}\mathrel{=}{}\<[E]%
\\
\>[4]{}\hsindent{3}{}\<[7]%
\>[7]{}\mathbf{case}\;\Varid{node}\;\mathbf{of}{}\<[E]%
\\
\>[7]{}\hsindent{3}{}\<[10]%
\>[10]{}\Conid{PdAtomBox}\;{}\<[21]%
\>[21]{}\Varid{atom}{}\<[30]%
\>[30]{}\to \Conid{PdNodeState}\;(\Varid{emptyInlets}\;\mathrm{1})\;{}\<[64]%
\>[64]{}[\mskip1.5mu \Varid{atom}\mskip1.5mu]{}\<[E]%
\\
\>[7]{}\hsindent{3}{}\<[10]%
\>[10]{}\Conid{PdObj}\;{}\<[21]%
\>[21]{}\anonymous \;\Varid{inl}\;\anonymous {}\<[30]%
\>[30]{}\to \Conid{PdNodeState}\;(\Varid{emptyInlets}\;\Varid{inl})\;{}\<[64]%
\>[64]{}[\mskip1.5mu \mskip1.5mu]{}\<[E]%
\\
\>[7]{}\hsindent{3}{}\<[10]%
\>[10]{}\Conid{PdMsgBox}\;{}\<[21]%
\>[21]{}\anonymous {}\<[30]%
\>[30]{}\to \Conid{PdNodeState}\;(\Varid{emptyInlets}\;\mathrm{1})\;{}\<[64]%
\>[64]{}[\mskip1.5mu \mskip1.5mu]{}\<[E]%
\ColumnHook
\end{hscode}\resethooks

\subsection{Operations}

The graphical language of Pure Data is graph-based and contains only nodes and
edges. The contents of nodes (object boxes, message boxes and atom boxes) are
textual. Like there are two kinds of edges (message and audio), there are also
two kinds of objects. Audio-handling objects are identified by a \texttt{\textasciitilde{}}
suffix in their names (the Pure Data documentation calls them ``tilde
objects''. In our interpreter, plain objects are implemented in the
\ensuremath{\Varid{sendMsg}} function (Section~\ref{msgs}) and tilde objects are
implemented in the \ensuremath{\Varid{performDsp}} function (Section~\ref{dsps}).

For printing to the log, we present a simple auxiliary function that adds to
the output log of the state value.

\begin{hscode}\SaveRestoreHook
\column{B}{@{}>{\hspre}l<{\hspost}@{}}%
\column{4}{@{}>{\hspre}l<{\hspost}@{}}%
\column{E}{@{}>{\hspre}l<{\hspost}@{}}%
\>[B]{}\Varid{printOut}\mathbin{::}[\mskip1.5mu \Conid{PdAtom}\mskip1.5mu]\to \Conid{PdState}\to \Conid{PdState}{}\<[E]%
\\
\>[B]{}\Varid{printOut}\;\Varid{atoms}\;\Varid{s}\mathrel{=}{}\<[E]%
\\
\>[B]{}\hsindent{4}{}\<[4]%
\>[4]{}\Varid{s}\;\{\mskip1.5mu \Varid{sLog}\mathrel{=}(\Varid{sLog}\;\Varid{s})\plus [\mskip1.5mu \Varid{intercalate}\;\text{\tt \char34 ~\char34}\mathbin{\$}\Varid{map}\;\Varid{show}\;\Varid{atoms}\mskip1.5mu]\mskip1.5mu\}{}\<[E]%
\ColumnHook
\end{hscode}\resethooks

The implementation of all non-audio nodes is done in the \ensuremath{\Varid{sendMsg}}
function, which pattern-matches on the structure of the node (which includes
the parsed representation of its textual definition). 

\begin{hscode}\SaveRestoreHook
\column{B}{@{}>{\hspre}l<{\hspost}@{}}%
\column{13}{@{}>{\hspre}l<{\hspost}@{}}%
\column{E}{@{}>{\hspre}l<{\hspost}@{}}%
\>[B]{}\Varid{sendMsg}\mathbin{::}{}\<[13]%
\>[13]{}\Conid{PdNode}\to [\mskip1.5mu \Conid{PdAtom}\mskip1.5mu]\to \Conid{Int}\to \Conid{PdNodeState}{}\<[E]%
\\
\>[13]{}\to (\Conid{PdNodeState},[\mskip1.5mu \Conid{String}\mskip1.5mu],[\mskip1.5mu [\mskip1.5mu \Conid{PdAtom}\mskip1.5mu]\mskip1.5mu],[\mskip1.5mu \Conid{PdEvent}\mskip1.5mu]){}\<[E]%
\ColumnHook
\end{hscode}\resethooks

Unlike the \ensuremath{\Varid{runCommand}} function used in the firing of message boxes,
which causes global effects to the graph evaluation (via indirect connections)
and therefore needs access to the whole state, \ensuremath{\Varid{sendMsg}} accesses
only the node's private state, producing a triple containing the new private
node state, any text produced for the output log, a list of messages to
be sent via the node's outlets and any new events to be scheduled.

Similarly to \ensuremath{\Varid{sendMsg}}, we define a single function that performs the
operations for all audio-processing objects:

\begin{hscode}\SaveRestoreHook
\column{B}{@{}>{\hspre}l<{\hspost}@{}}%
\column{E}{@{}>{\hspre}l<{\hspost}@{}}%
\>[B]{}\Varid{performDsp}\mathbin{::}\Conid{PdNode}\to \Conid{PdNodeState}\to ([\mskip1.5mu [\mskip1.5mu \Conid{PdAtom}\mskip1.5mu]\mskip1.5mu],[\mskip1.5mu \Conid{PdAtom}\mskip1.5mu]){}\<[E]%
\ColumnHook
\end{hscode}\resethooks

The \ensuremath{\Varid{performDsp}} function takes the object, its node state and
outputs the audio buffer to be sent at the node's outlets, and the
updated internal data for the node.

We did not implement the full range of objects supported by Pure Data since
our goal was not to produce a full-fledged computer music application, but
we included a few representative objects that allow us to demonstrate the
interpreter and the various behaviors of objects.

\subsubsection{Atom boxes}

When given a float, atom boxes update their internal memory and propagate the
value. When given a \texttt{bang}, they just propagate the value.

\begin{hscode}\SaveRestoreHook
\column{B}{@{}>{\hspre}l<{\hspost}@{}}%
\column{4}{@{}>{\hspre}l<{\hspost}@{}}%
\column{E}{@{}>{\hspre}l<{\hspost}@{}}%
\>[B]{}\Varid{sendMsg}\;(\Conid{PdAtomBox}\;\anonymous )\;(\Conid{PdSymbol}\;\text{\tt \char34 float\char34}\mathbin{:}\Varid{fl})\;\mathrm{0}\;\anonymous \mathrel{=}{}\<[E]%
\\
\>[B]{}\hsindent{4}{}\<[4]%
\>[4]{}(\Conid{PdNodeState}\;(\Varid{fromList}\;[\mskip1.5mu \mskip1.5mu])\;\Varid{fl},[\mskip1.5mu \mskip1.5mu],[\mskip1.5mu \Conid{PdSymbol}\;\text{\tt \char34 float\char34}\mathbin{:}\Varid{fl}\mskip1.5mu],[\mskip1.5mu \mskip1.5mu]){}\<[E]%
\\[\blanklineskip]%
\>[B]{}\Varid{sendMsg}\;(\Conid{PdAtomBox}\;\anonymous )\;[\mskip1.5mu \Conid{PdSymbol}\;\text{\tt \char34 bang\char34}\mskip1.5mu]\;\mathrm{0}\;\Varid{ns}\mathord{@}(\Conid{PdNodeState}\;\anonymous \;\Varid{mem})\mathrel{=}{}\<[E]%
\\
\>[B]{}\hsindent{4}{}\<[4]%
\>[4]{}(\Varid{ns},[\mskip1.5mu \mskip1.5mu],[\mskip1.5mu \Conid{PdSymbol}\;\text{\tt \char34 float\char34}\mathbin{:}\Varid{mem}\mskip1.5mu],[\mskip1.5mu \mskip1.5mu]){}\<[E]%
\ColumnHook
\end{hscode}\resethooks

\subsubsection{An object with side-effects: \texttt{print}}

The \texttt{print} object accepts data through its inlet and prints it to
the log console. It demonstrates the use of the log console as a global
side-effect.

\begin{hscode}\SaveRestoreHook
\column{B}{@{}>{\hspre}l<{\hspost}@{}}%
\column{4}{@{}>{\hspre}l<{\hspost}@{}}%
\column{E}{@{}>{\hspre}l<{\hspost}@{}}%
\>[B]{}\Varid{sendMsg}\;(\Conid{PdObj}\;(\Conid{PdSymbol}\;\text{\tt \char34 print\char34}\mathbin{:}\Varid{xs})\;\anonymous \;\anonymous )\;(\Conid{PdSymbol}\;\text{\tt \char34 float\char34}\mathbin{:}\Varid{fs})\;\mathrm{0}\;\Varid{ns}\mathrel{=}{}\<[E]%
\\
\>[B]{}\hsindent{4}{}\<[4]%
\>[4]{}(\Varid{ns},[\mskip1.5mu \text{\tt \char34 print:~\char34}\plus (\Varid{intercalate}\;\text{\tt \char34 ~\char34}\mathbin{\$}\Varid{map}\;\Varid{show}\;(\Varid{xs}\plus \Varid{fs}))\mskip1.5mu],[\mskip1.5mu \mskip1.5mu],[\mskip1.5mu \mskip1.5mu]){}\<[E]%
\\[\blanklineskip]%
\>[B]{}\Varid{sendMsg}\;(\Conid{PdObj}\;(\Conid{PdSymbol}\;\text{\tt \char34 print\char34}\mathbin{:}\Varid{xs})\;\anonymous \;\anonymous )\;(\Conid{PdSymbol}\;\text{\tt \char34 list\char34}\mathbin{:}\Varid{ls})\;\mathrm{0}\;\Varid{ns}\mathrel{=}{}\<[E]%
\\
\>[B]{}\hsindent{4}{}\<[4]%
\>[4]{}(\Varid{ns},[\mskip1.5mu \text{\tt \char34 print:~\char34}\plus (\Varid{intercalate}\;\text{\tt \char34 ~\char34}\mathbin{\$}\Varid{map}\;\Varid{show}\;(\Varid{xs}\plus \Varid{ls}))\mskip1.5mu],[\mskip1.5mu \mskip1.5mu],[\mskip1.5mu \mskip1.5mu]){}\<[E]%
\\[\blanklineskip]%
\>[B]{}\Varid{sendMsg}\;(\Conid{PdObj}\;(\Conid{PdSymbol}\;\text{\tt \char34 print\char34}\mathbin{:}\Varid{xs})\;\anonymous \;\anonymous )\;\Varid{atoms}\;\mathrm{0}\;\Varid{ns}\mathrel{=}{}\<[E]%
\\
\>[B]{}\hsindent{4}{}\<[4]%
\>[4]{}(\Varid{ns},[\mskip1.5mu \text{\tt \char34 print:~\char34}\plus (\Varid{intercalate}\;\text{\tt \char34 ~\char34}\mathbin{\$}\Varid{map}\;\Varid{show}\;\Varid{atoms})\mskip1.5mu],[\mskip1.5mu \mskip1.5mu],[\mskip1.5mu \mskip1.5mu]){}\<[E]%
\ColumnHook
\end{hscode}\resethooks

\subsubsection{An object with hot and cold inlets: \texttt{+}}

In Pure Data, the first inlet of a node is the ``hot'' inlet; when data is
received through it, the action of the node is performed. When data arrives in
``cold'' inlets, it stays queued until the ``hot'' inlet causes the object to
be evaluated.

The \texttt{+} object demonstrates the behavior of hot and cold inlets. 
When a number arrives in the hot inlet, it sums the values in inlets 0 and 1
and sends it through its outlet. When a \texttt{bang} arrives in the hot outlet,
the most recently received values in the inlet buffers are used for the sum instead.

\begin{hscode}\SaveRestoreHook
\column{B}{@{}>{\hspre}l<{\hspost}@{}}%
\column{4}{@{}>{\hspre}l<{\hspost}@{}}%
\column{7}{@{}>{\hspre}l<{\hspost}@{}}%
\column{10}{@{}>{\hspre}l<{\hspost}@{}}%
\column{14}{@{}>{\hspre}l<{\hspost}@{}}%
\column{E}{@{}>{\hspre}l<{\hspost}@{}}%
\>[B]{}\Varid{sendMsg}\;{}\<[10]%
\>[10]{}(\Conid{PdObj}\;[\mskip1.5mu \Conid{PdSymbol}\;\text{\tt \char34 +\char34},\Varid{n}\mskip1.5mu]\;\anonymous \;\anonymous )\;[\mskip1.5mu \Conid{PdSymbol}\;\text{\tt \char34 float\char34},\Varid{fl}\mskip1.5mu]\;\mathrm{0}\;{}\<[E]%
\\
\>[10]{}\hsindent{4}{}\<[14]%
\>[14]{}(\Conid{PdNodeState}\;\Varid{ins}\;\Varid{mem})\mathrel{=}{}\<[E]%
\\
\>[B]{}\hsindent{4}{}\<[4]%
\>[4]{}\mathbf{let}{}\<[E]%
\\
\>[4]{}\hsindent{3}{}\<[7]%
\>[7]{}(\Conid{PdFloat}\;val_0)\mathrel{=}\Varid{fl}{}\<[E]%
\\
\>[4]{}\hsindent{3}{}\<[7]%
\>[7]{}inlet_1\mathrel{=}\Varid{index}\;\Varid{ins}\;\mathrm{1}{}\<[E]%
\\
\>[4]{}\hsindent{3}{}\<[7]%
\>[7]{}(\Conid{PdFloat}\;val_1)\mathrel{=}\mathbf{if}\;inlet_1\equiv [\mskip1.5mu \mskip1.5mu]\;\mathbf{then}\;\Varid{n}\;\mathbf{else}\;\Varid{head}\;inlet_1{}\<[E]%
\\
\>[4]{}\hsindent{3}{}\<[7]%
\>[7]{}\Varid{mem'}\mathrel{=}[\mskip1.5mu \Conid{PdFloat}\;(val_0\mathbin{+}val_1)\mskip1.5mu]{}\<[E]%
\\
\>[4]{}\hsindent{3}{}\<[7]%
\>[7]{}\Varid{ns'}\mathrel{=}\Conid{PdNodeState}\;(\Varid{update}\;\mathrm{0}\;[\mskip1.5mu \Varid{fl}\mskip1.5mu]\;\Varid{ins})\;\Varid{mem'}{}\<[E]%
\\
\>[B]{}\hsindent{4}{}\<[4]%
\>[4]{}\mathbf{in}{}\<[E]%
\\
\>[4]{}\hsindent{3}{}\<[7]%
\>[7]{}(\Varid{ns'},[\mskip1.5mu \mskip1.5mu],[\mskip1.5mu \Conid{PdSymbol}\;\text{\tt \char34 float\char34}\mathbin{:}\Varid{mem'}\mskip1.5mu],[\mskip1.5mu \mskip1.5mu]){}\<[E]%
\\[\blanklineskip]%
\>[B]{}\Varid{sendMsg}\;{}\<[10]%
\>[10]{}(\Conid{PdObj}\;[\mskip1.5mu \Conid{PdSymbol}\;\text{\tt \char34 +\char34},\Varid{n}\mskip1.5mu]\;\anonymous \;\anonymous )\;[\mskip1.5mu \Conid{PdSymbol}\;\text{\tt \char34 bang\char34}\mskip1.5mu]\;\mathrm{0}\;{}\<[E]%
\\
\>[10]{}\hsindent{4}{}\<[14]%
\>[14]{}(\Conid{PdNodeState}\;\Varid{ins}\;\Varid{mem})\mathrel{=}{}\<[E]%
\\
\>[B]{}\hsindent{4}{}\<[4]%
\>[4]{}\mathbf{let}{}\<[E]%
\\
\>[4]{}\hsindent{3}{}\<[7]%
\>[7]{}inlet_0\mathrel{=}\Varid{index}\;\Varid{ins}\;\mathrm{0}{}\<[E]%
\\
\>[4]{}\hsindent{3}{}\<[7]%
\>[7]{}(\Conid{PdFloat}\;val_0)\mathrel{=}\mathbf{if}\;inlet_0\equiv [\mskip1.5mu \mskip1.5mu]\;\mathbf{then}\;(\Conid{PdFloat}\;\mathrm{0})\;\mathbf{else}\;\Varid{head}\;inlet_0{}\<[E]%
\\
\>[4]{}\hsindent{3}{}\<[7]%
\>[7]{}inlet_1\mathrel{=}\Varid{index}\;\Varid{ins}\;\mathrm{1}{}\<[E]%
\\
\>[4]{}\hsindent{3}{}\<[7]%
\>[7]{}(\Conid{PdFloat}\;val_1)\mathrel{=}\mathbf{if}\;inlet_1\equiv [\mskip1.5mu \mskip1.5mu]\;\mathbf{then}\;\Varid{n}\;\mathbf{else}\;\Varid{head}\;inlet_1{}\<[E]%
\\
\>[4]{}\hsindent{3}{}\<[7]%
\>[7]{}\Varid{mem'}\mathrel{=}[\mskip1.5mu \Conid{PdFloat}\;(val_0\mathbin{+}val_1)\mskip1.5mu]{}\<[E]%
\\
\>[4]{}\hsindent{3}{}\<[7]%
\>[7]{}\Varid{ns'}\mathrel{=}\Conid{PdNodeState}\;\Varid{ins}\;\Varid{mem'}{}\<[E]%
\\
\>[B]{}\hsindent{4}{}\<[4]%
\>[4]{}\mathbf{in}{}\<[E]%
\\
\>[4]{}\hsindent{3}{}\<[7]%
\>[7]{}(\Varid{ns'},[\mskip1.5mu \mskip1.5mu],[\mskip1.5mu \Conid{PdSymbol}\;\text{\tt \char34 float\char34}\mathbin{:}\Varid{mem'}\mskip1.5mu],[\mskip1.5mu \mskip1.5mu]){}\<[E]%
\ColumnHook
\end{hscode}\resethooks

\subsubsection{Objects producing timed events: \text{\tt delay} and \text{\tt metro}}

The \texttt{delay} object demonstrates how objects generate future events.
We handle four cases: receiving a \texttt{bang} message schedules a
\texttt{tick} event. When received, it outputs a \texttt{bang} to the 
node's outlets.

\begin{hscode}\SaveRestoreHook
\column{B}{@{}>{\hspre}l<{\hspost}@{}}%
\column{4}{@{}>{\hspre}l<{\hspost}@{}}%
\column{E}{@{}>{\hspre}l<{\hspost}@{}}%
\>[B]{}\Varid{sendMsg}\;(\Conid{PdObj}\;[\mskip1.5mu \Conid{PdSymbol}\;\text{\tt \char34 delay\char34},\Conid{PdFloat}\;\Varid{time}\mskip1.5mu]\;\Varid{inl}\;\anonymous )\;[\mskip1.5mu \Conid{PdSymbol}\;\text{\tt \char34 bang\char34}\mskip1.5mu]\;\mathrm{0}\;\Varid{ns}\mathrel{=}{}\<[E]%
\\
\>[B]{}\hsindent{4}{}\<[4]%
\>[4]{}(\Varid{ns},[\mskip1.5mu \mskip1.5mu],[\mskip1.5mu \mskip1.5mu],[\mskip1.5mu \Conid{PdEvent}\;(\Varid{floor}\;\Varid{time})\;\mathrm{0}\;[\mskip1.5mu \Conid{PdSymbol}\;\text{\tt \char34 tick\char34}\mskip1.5mu]\mskip1.5mu]){}\<[E]%
\\[\blanklineskip]%
\>[B]{}\Varid{sendMsg}\;(\Conid{PdObj}\;(\Conid{PdSymbol}\;\text{\tt \char34 delay\char34}\mathbin{:}\Varid{t})\;\Varid{inl}\;\anonymous )\;[\mskip1.5mu \Conid{PdSymbol}\;\text{\tt \char34 tick\char34}\mskip1.5mu]\;\mathrm{0}\;\Varid{ns}\mathrel{=}{}\<[E]%
\\
\>[B]{}\hsindent{4}{}\<[4]%
\>[4]{}(\Varid{ns},[\mskip1.5mu \mskip1.5mu],[\mskip1.5mu [\mskip1.5mu \Conid{PdSymbol}\;\text{\tt \char34 bang\char34}\mskip1.5mu]\mskip1.5mu],[\mskip1.5mu \mskip1.5mu]){}\<[E]%
\ColumnHook
\end{hscode}\resethooks

The \texttt{metro} node, on its turn, expands on the \texttt{delay}
functionality, implementing a metronome: it sends a series of \texttt{bang}
messages at regular time intervals. It also has a second inlet which allows
updating the interval.

We handle four cases: receiving a \texttt{bang} message to start the
metronome, receiving a \texttt{stop} message to stop it, and receiving the
internally-scheduled \texttt{tick} when the metronome is either on or off.

\begin{hscode}\SaveRestoreHook
\column{B}{@{}>{\hspre}l<{\hspost}@{}}%
\column{4}{@{}>{\hspre}l<{\hspost}@{}}%
\column{7}{@{}>{\hspre}l<{\hspost}@{}}%
\column{10}{@{}>{\hspre}l<{\hspost}@{}}%
\column{14}{@{}>{\hspre}l<{\hspost}@{}}%
\column{E}{@{}>{\hspre}l<{\hspost}@{}}%
\>[B]{}\Varid{sendMsg}\;{}\<[10]%
\>[10]{}(\Conid{PdObj}\;(\Conid{PdSymbol}\;\text{\tt \char34 metro\char34}\mathbin{:}\Varid{xs})\;\Varid{inl}\;\anonymous )\;[\mskip1.5mu \Conid{PdSymbol}\;\text{\tt \char34 bang\char34}\mskip1.5mu]\;\mathrm{0}\;{}\<[E]%
\\
\>[10]{}\hsindent{4}{}\<[14]%
\>[14]{}(\Conid{PdNodeState}\;\Varid{ins}\;\Varid{mem})\mathrel{=}{}\<[E]%
\\
\>[B]{}\hsindent{4}{}\<[4]%
\>[4]{}\mathbf{let}{}\<[E]%
\\
\>[4]{}\hsindent{3}{}\<[7]%
\>[7]{}inlet_1\mathrel{=}\Varid{index}\;\Varid{ins}\;\mathrm{1}{}\<[E]%
\\
\>[4]{}\hsindent{3}{}\<[7]%
\>[7]{}(\Conid{PdFloat}\;\Varid{time})\mathrel{=}\Varid{head}\;(inlet_1\plus \Varid{mem}\plus \Varid{xs}\plus [\mskip1.5mu \Conid{PdFloat}\;\mathrm{1000}\mskip1.5mu]){}\<[E]%
\\
\>[4]{}\hsindent{3}{}\<[7]%
\>[7]{}\Varid{ns'}\mathrel{=}\Conid{PdNodeState}\;(\Varid{emptyInlets}\;\Varid{inl})\;[\mskip1.5mu \Conid{PdFloat}\;\Varid{time},\Conid{PdSymbol}\;\text{\tt \char34 on\char34}\mskip1.5mu]{}\<[E]%
\\
\>[B]{}\hsindent{4}{}\<[4]%
\>[4]{}\mathbf{in}{}\<[E]%
\\
\>[4]{}\hsindent{3}{}\<[7]%
\>[7]{}(\Varid{ns'},[\mskip1.5mu \mskip1.5mu],[\mskip1.5mu [\mskip1.5mu \Conid{PdSymbol}\;\text{\tt \char34 bang\char34}\mskip1.5mu]\mskip1.5mu],[\mskip1.5mu \Conid{PdEvent}\;(\Varid{floor}\;\Varid{time})\;\mathrm{0}\;[\mskip1.5mu \Conid{PdSymbol}\;\text{\tt \char34 tick\char34}\mskip1.5mu]\mskip1.5mu]){}\<[E]%
\\[\blanklineskip]%
\>[B]{}\Varid{sendMsg}\;{}\<[10]%
\>[10]{}(\Conid{PdObj}\;(\Conid{PdSymbol}\;\text{\tt \char34 metro\char34}\mathbin{:}\Varid{xs})\;\Varid{inl}\;\anonymous )\;[\mskip1.5mu \Conid{PdSymbol}\;\text{\tt \char34 stop\char34}\mskip1.5mu]\;\mathrm{0}\;{}\<[E]%
\\
\>[10]{}\hsindent{4}{}\<[14]%
\>[14]{}(\Conid{PdNodeState}\;\Varid{ins}\;[\mskip1.5mu \Conid{PdFloat}\;\Varid{time},\Conid{PdSymbol}\;\text{\tt \char34 on\char34}\mskip1.5mu])\mathrel{=}{}\<[E]%
\\
\>[B]{}\hsindent{4}{}\<[4]%
\>[4]{}(\Conid{PdNodeState}\;\Varid{ins}\;[\mskip1.5mu \Conid{PdFloat}\;\Varid{time},\Conid{PdSymbol}\;\text{\tt \char34 off\char34}\mskip1.5mu],[\mskip1.5mu \mskip1.5mu],[\mskip1.5mu \mskip1.5mu],[\mskip1.5mu \mskip1.5mu]){}\<[E]%
\\[\blanklineskip]%
\>[B]{}\Varid{sendMsg}\;{}\<[10]%
\>[10]{}(\Conid{PdObj}\;(\Conid{PdSymbol}\;\text{\tt \char34 metro\char34}\mathbin{:}\Varid{xs})\;\Varid{inl}\;\anonymous )\;[\mskip1.5mu \Conid{PdSymbol}\;\text{\tt \char34 tick\char34}\mskip1.5mu]\;\mathrm{0}\;{}\<[E]%
\\
\>[10]{}\hsindent{4}{}\<[14]%
\>[14]{}\Varid{ns}\mathord{@}(\Conid{PdNodeState}\;\Varid{ins}\;[\mskip1.5mu \Conid{PdFloat}\;\Varid{time},\Conid{PdSymbol}\;\text{\tt \char34 on\char34}\mskip1.5mu])\mathrel{=}{}\<[E]%
\\
\>[B]{}\hsindent{4}{}\<[4]%
\>[4]{}(\Varid{ns},[\mskip1.5mu \mskip1.5mu],[\mskip1.5mu [\mskip1.5mu \Conid{PdSymbol}\;\text{\tt \char34 bang\char34}\mskip1.5mu]\mskip1.5mu],[\mskip1.5mu \Conid{PdEvent}\;(\Varid{floor}\;\Varid{time})\;\mathrm{0}\;[\mskip1.5mu \Conid{PdSymbol}\;\text{\tt \char34 tick\char34}\mskip1.5mu]\mskip1.5mu]){}\<[E]%
\\[\blanklineskip]%
\>[B]{}\Varid{sendMsg}\;{}\<[10]%
\>[10]{}(\Conid{PdObj}\;(\Conid{PdSymbol}\;\text{\tt \char34 metro\char34}\mathbin{:}\Varid{xs})\;\Varid{inl}\;\anonymous )\;[\mskip1.5mu \Conid{PdSymbol}\;\text{\tt \char34 tick\char34}\mskip1.5mu]\;\mathrm{0}\;{}\<[E]%
\\
\>[10]{}\hsindent{4}{}\<[14]%
\>[14]{}\Varid{ns}\mathord{@}(\Conid{PdNodeState}\;\Varid{ins}\;[\mskip1.5mu \anonymous ,\Conid{PdSymbol}\;\text{\tt \char34 off\char34}\mskip1.5mu])\mathrel{=}{}\<[E]%
\\
\>[B]{}\hsindent{4}{}\<[4]%
\>[4]{}(\Varid{ns},[\mskip1.5mu \mskip1.5mu],[\mskip1.5mu \mskip1.5mu],[\mskip1.5mu \mskip1.5mu]){}\<[E]%
\ColumnHook
\end{hscode}\resethooks

\subsubsection{Message handlers for audio objects: \text{\tt osc\char126{}} and \text{\tt line\char126{}}}
\label{linemsg}

Some audio objects in Pure Data also accept messages. The \texttt{osc\textasciitilde{}} object
implements a sinewave oscillator. Sending a float to it, we configure its
frequency, which is stored in the node's internal memory. Note that the actual
oscillator is not implemented here, but in the DSP handler for this object
type in function \ensuremath{\Varid{performDsp}}, in Section~\ref{oscdsp}.

\begin{hscode}\SaveRestoreHook
\column{B}{@{}>{\hspre}l<{\hspost}@{}}%
\column{4}{@{}>{\hspre}l<{\hspost}@{}}%
\column{10}{@{}>{\hspre}l<{\hspost}@{}}%
\column{14}{@{}>{\hspre}l<{\hspost}@{}}%
\column{E}{@{}>{\hspre}l<{\hspost}@{}}%
\>[B]{}\Varid{sendMsg}\;{}\<[10]%
\>[10]{}(\Conid{PdObj}\;(\Conid{PdSymbol}\;\text{\tt \char34 osc\char126 \char34}\mathbin{:}\anonymous )\;\anonymous \;\anonymous )\;[\mskip1.5mu \Conid{PdSymbol}\;\text{\tt \char34 float\char34},\Conid{PdFloat}\;\Varid{freq}\mskip1.5mu]\;\mathrm{0}\;{}\<[E]%
\\
\>[10]{}\hsindent{4}{}\<[14]%
\>[14]{}(\Conid{PdNodeState}\;\Varid{ins}\;[\mskip1.5mu \anonymous ,\Varid{position}\mskip1.5mu])\mathrel{=}{}\<[E]%
\\
\>[B]{}\hsindent{4}{}\<[4]%
\>[4]{}(\Conid{PdNodeState}\;\Varid{ins}\;[\mskip1.5mu \Conid{PdFloat}\;((\mathrm{2}\mathbin{*}\Varid{pi})\mathbin{/}(\mathrm{32000}\mathbin{/}\Varid{freq})),\Varid{position}\mskip1.5mu],[\mskip1.5mu \mskip1.5mu],[\mskip1.5mu \mskip1.5mu],[\mskip1.5mu \mskip1.5mu]){}\<[E]%
\ColumnHook
\end{hscode}\resethooks

The \texttt{line\textasciitilde{}} object implements a linear function over time. It can be used,
for example, to implement gradual changes of frequency or amplitude. Its internal
memory stores values $current$, $target$ and $delta$. It accepts a message with
two floats, indicating the new target value and the time interval to take ramping
from the current value to the new target.

\begin{hscode}\SaveRestoreHook
\column{B}{@{}>{\hspre}l<{\hspost}@{}}%
\column{4}{@{}>{\hspre}l<{\hspost}@{}}%
\column{7}{@{}>{\hspre}l<{\hspost}@{}}%
\column{10}{@{}>{\hspre}l<{\hspost}@{}}%
\column{14}{@{}>{\hspre}l<{\hspost}@{}}%
\column{E}{@{}>{\hspre}l<{\hspost}@{}}%
\>[B]{}\Varid{sendMsg}\;{}\<[10]%
\>[10]{}(\Conid{PdObj}\;[\mskip1.5mu \Conid{PdSymbol}\;\text{\tt \char34 line\char126 \char34}\mskip1.5mu]\;\anonymous \;\anonymous )\;{}\<[E]%
\\
\>[10]{}\hsindent{4}{}\<[14]%
\>[14]{}[\mskip1.5mu \Conid{PdSymbol}\;\text{\tt \char34 list\char34},\Conid{PdFloat}\;\Varid{amp},\Conid{PdFloat}\;\Varid{time}\mskip1.5mu]\;\mathrm{0}\;{}\<[E]%
\\
\>[10]{}\hsindent{4}{}\<[14]%
\>[14]{}(\Conid{PdNodeState}\;\Varid{ins}\;\Varid{mem})\mathrel{=}{}\<[E]%
\\
\>[B]{}\hsindent{4}{}\<[4]%
\>[4]{}\mathbf{let}{}\<[E]%
\\
\>[4]{}\hsindent{3}{}\<[7]%
\>[7]{}[\mskip1.5mu \Conid{PdFloat}\;\Varid{current},\Conid{PdFloat}\;\Varid{target},\Conid{PdFloat}\;\Varid{delta}\mskip1.5mu]\mathrel{=}{}\<[E]%
\\
\>[7]{}\hsindent{3}{}\<[10]%
\>[10]{}\mathbf{if}\;\Varid{mem}\not\equiv [\mskip1.5mu \mskip1.5mu]\;\mathbf{then}\;\Varid{mem}\;\mathbf{else}\;[\mskip1.5mu \Conid{PdFloat}\;\mathrm{0},\Conid{PdFloat}\;\mathrm{0},\Conid{PdFloat}\;\mathrm{0}\mskip1.5mu]{}\<[E]%
\\
\>[4]{}\hsindent{3}{}\<[7]%
\>[7]{}\Varid{mem'}\mathrel{=}{}\<[E]%
\\
\>[7]{}\hsindent{3}{}\<[10]%
\>[10]{}[\mskip1.5mu \Conid{PdFloat}\;\Varid{current},\Conid{PdFloat}\;\Varid{amp},\Conid{PdFloat}\;((\Varid{amp}\mathbin{-}\Varid{current})\mathbin{/}(\Varid{time}\mathbin{*}\mathrm{32}))\mskip1.5mu]{}\<[E]%
\\
\>[B]{}\hsindent{4}{}\<[4]%
\>[4]{}\mathbf{in}{}\<[E]%
\\
\>[4]{}\hsindent{3}{}\<[7]%
\>[7]{}(\Conid{PdNodeState}\;\Varid{ins}\;\Varid{mem'},[\mskip1.5mu \mskip1.5mu],[\mskip1.5mu \mskip1.5mu],[\mskip1.5mu \mskip1.5mu]){}\<[E]%
\ColumnHook
\end{hscode}\resethooks

\subsubsection{Cold inlets}

Since cold inlets are passive and only store the incoming data in the inlet
buffer without executing any node-specific operation, the implementation for
cold inlets can be shared by all types of node.

\begin{hscode}\SaveRestoreHook
\column{B}{@{}>{\hspre}l<{\hspost}@{}}%
\column{4}{@{}>{\hspre}l<{\hspost}@{}}%
\column{E}{@{}>{\hspre}l<{\hspost}@{}}%
\>[B]{}\Varid{sendMsg}\;\Varid{node}\;(\Conid{PdSymbol}\;\text{\tt \char34 float\char34}\mathbin{:}\Varid{fs})\;\Varid{inl}\;(\Conid{PdNodeState}\;\Varid{ins}\;\Varid{mem})\mid \Varid{inl}\mathbin{>}\mathrm{0}\mathrel{=}{}\<[E]%
\\
\>[B]{}\hsindent{4}{}\<[4]%
\>[4]{}(\Conid{PdNodeState}\;(\Varid{update}\;\Varid{inl}\;\Varid{fs}\;\Varid{ins})\;\Varid{mem},[\mskip1.5mu \mskip1.5mu],[\mskip1.5mu \mskip1.5mu],[\mskip1.5mu \mskip1.5mu]){}\<[E]%
\\[\blanklineskip]%
\>[B]{}\Varid{sendMsg}\;\Varid{node}\;\Varid{atoms}\;\Varid{inl}\;(\Conid{PdNodeState}\;\Varid{ins}\;\Varid{mem})\mid \Varid{inl}\mathbin{>}\mathrm{0}\mathrel{=}{}\<[E]%
\\
\>[B]{}\hsindent{4}{}\<[4]%
\>[4]{}(\Conid{PdNodeState}\;(\Varid{update}\;\Varid{inl}\;\Varid{atoms}\;\Varid{ins})\;\Varid{mem},[\mskip1.5mu \mskip1.5mu],[\mskip1.5mu \mskip1.5mu],[\mskip1.5mu \mskip1.5mu]){}\<[E]%
\ColumnHook
\end{hscode}\resethooks

\subsubsection{Data objects: \texttt{float} and \texttt{list}}

The \texttt{float} and \texttt{list} objects store and forward data of their
respective types. They have two inlets for accepting new data. When given data
through its first inlet, the object stores it in its internal memory and
outputs the value through the outlet. When given data through its second
inlet, it only stores the value. When given a unit event (called \texttt{bang}
in Pure Data), it outputs the most recently received value (or the one given
in its creation argument, or zero as a fallback).

\begin{hscode}\SaveRestoreHook
\column{B}{@{}>{\hspre}l<{\hspost}@{}}%
\column{4}{@{}>{\hspre}l<{\hspost}@{}}%
\column{7}{@{}>{\hspre}l<{\hspost}@{}}%
\column{12}{@{}>{\hspre}l<{\hspost}@{}}%
\column{E}{@{}>{\hspre}l<{\hspost}@{}}%
\>[B]{}\Varid{sendMsg}\;\Varid{cmd}\mathord{@}(\Conid{PdObj}\;(\Conid{PdSymbol}\;\text{\tt \char34 float\char34}\mathbin{:}\Varid{xs})\;\Varid{inl}\;\anonymous )\;\Varid{atoms}\;\mathrm{0}\;\Varid{ns}\mathrel{=}{}\<[E]%
\\
\>[B]{}\hsindent{4}{}\<[4]%
\>[4]{}\Varid{dataObject}\;\Varid{cmd}\;\Varid{atoms}\;\Varid{ns}{}\<[E]%
\\[\blanklineskip]%
\>[B]{}\Varid{sendMsg}\;\Varid{cmd}\mathord{@}(\Conid{PdObj}\;(\Conid{PdSymbol}\;\text{\tt \char34 list\char34}\mathbin{:}\Varid{xs})\;\Varid{inl}\;\anonymous )\;\Varid{atoms}\;\mathrm{0}\;\Varid{ns}\mathrel{=}{}\<[E]%
\\
\>[B]{}\hsindent{4}{}\<[4]%
\>[4]{}\Varid{dataObject}\;\Varid{cmd}\;\Varid{atoms}\;\Varid{ns}{}\<[E]%
\\[\blanklineskip]%
\>[B]{}\Varid{dataObject}\;(\Conid{PdObj}\;(\Conid{PdSymbol}\;\Varid{a}\mathbin{:}\Varid{xs})\;\Varid{inl}\;\anonymous )\;[\mskip1.5mu \Conid{PdSymbol}\;\text{\tt \char34 bang\char34}\mskip1.5mu]\;{}\<[E]%
\\
\>[B]{}\hsindent{12}{}\<[12]%
\>[12]{}(\Conid{PdNodeState}\;\Varid{ins}\;\Varid{mem})\mathrel{=}{}\<[E]%
\\
\>[B]{}\hsindent{4}{}\<[4]%
\>[4]{}\mathbf{let}{}\<[E]%
\\
\>[4]{}\hsindent{3}{}\<[7]%
\>[7]{}inlet_1\mathrel{=}\Varid{index}\;\Varid{ins}\;\mathrm{1}{}\<[E]%
\\
\>[4]{}\hsindent{3}{}\<[7]%
\>[7]{}\Conid{Just}\;\Varid{mem'}\mathrel{=}\Varid{find}\;(\not\equiv [\mskip1.5mu \mskip1.5mu])\;[\mskip1.5mu inlet_1,\Varid{mem},\Varid{xs},[\mskip1.5mu \Conid{PdFloat}\;\mathrm{0}\mskip1.5mu]\mskip1.5mu]{}\<[E]%
\\
\>[B]{}\hsindent{4}{}\<[4]%
\>[4]{}\mathbf{in}{}\<[E]%
\\
\>[4]{}\hsindent{3}{}\<[7]%
\>[7]{}(\Conid{PdNodeState}\;(\Varid{emptyInlets}\;\Varid{inl})\;\Varid{mem'},[\mskip1.5mu \mskip1.5mu],[\mskip1.5mu \Conid{PdSymbol}\;\Varid{a}\mathbin{:}\Varid{mem'}\mskip1.5mu],[\mskip1.5mu \mskip1.5mu]){}\<[E]%
\\[\blanklineskip]%
\>[B]{}\Varid{dataObject}\;(\Conid{PdObj}\;(\Conid{PdSymbol}\;\Varid{a}\mathbin{:}\Varid{xs})\;\Varid{inl}\;\anonymous )\;(\Conid{PdSymbol}\;\Varid{b}\mathbin{:}\Varid{fl})\;\anonymous \mid \Varid{a}\equiv \Varid{b}\mathrel{=}{}\<[E]%
\\
\>[B]{}\hsindent{4}{}\<[4]%
\>[4]{}(\Conid{PdNodeState}\;(\Varid{emptyInlets}\;\Varid{inl})\;\Varid{fl},[\mskip1.5mu \mskip1.5mu],[\mskip1.5mu \Conid{PdSymbol}\;\Varid{a}\mathbin{:}\Varid{fl}\mskip1.5mu],[\mskip1.5mu \mskip1.5mu]){}\<[E]%
\ColumnHook
\end{hscode}\resethooks

\subsubsection{Audio handling operations: \text{\tt osc\char126{}}, \text{\tt line\char126{}} and \text{\tt \char42{}\char126{}}}
\label{dsps}
\label{oscdsp}

Audio handling is performed by function \ensuremath{\Varid{performDsp}}, which implements
cases for each type of audio object.

Object \texttt{osc\textasciitilde{}} is the sinewave oscillator. It holds two values in its
internal memory, \ensuremath{\Varid{delta}} and \ensuremath{\Varid{position}}, through which a wave describing a
sine function is incrementally computed.

We handle two cases here: when the internal memory is empty, the parameters
are initialized according to the \ensuremath{\Varid{freq}} creation argument; when the 
memory is initialized, we produce the new buffer calculating
64 new values, determine the next position to start the wave in the next
iteration, store this value in the internal memory, and output the buffer
through the node's outlet.

\begin{hscode}\SaveRestoreHook
\column{B}{@{}>{\hspre}l<{\hspost}@{}}%
\column{4}{@{}>{\hspre}l<{\hspost}@{}}%
\column{7}{@{}>{\hspre}l<{\hspost}@{}}%
\column{13}{@{}>{\hspre}l<{\hspost}@{}}%
\column{E}{@{}>{\hspre}l<{\hspost}@{}}%
\>[B]{}\Varid{performDsp}\;\Varid{obj}\mathord{@}(\Conid{PdObj}\;[\mskip1.5mu \Conid{PdSymbol}\;\text{\tt \char34 osc\char126 \char34},\Conid{PdFloat}\;\Varid{freq}\mskip1.5mu]\;\anonymous \;\anonymous )\;(\Conid{PdNodeState}\;\Varid{ins}\;[\mskip1.5mu \mskip1.5mu])\mathrel{=}{}\<[E]%
\\
\>[B]{}\hsindent{4}{}\<[4]%
\>[4]{}\Varid{performDsp}\;\Varid{obj}\;(\Conid{PdNodeState}\;\Varid{ins}\;[\mskip1.5mu \Conid{PdFloat}\;((\mathrm{2}\mathbin{*}\Varid{pi})\mathbin{/}(\mathrm{32000}\mathbin{/}\Varid{freq})),\Conid{PdFloat}\;\mathrm{0}\mskip1.5mu]){}\<[E]%
\\[\blanklineskip]%
\>[B]{}\Varid{performDsp}\;{}\<[13]%
\>[13]{}(\Conid{PdObj}\;[\mskip1.5mu \Conid{PdSymbol}\;\text{\tt \char34 osc\char126 \char34},\anonymous \mskip1.5mu]\;\anonymous \;\anonymous )\;{}\<[E]%
\\
\>[13]{}(\Conid{PdNodeState}\;\Varid{ins}\;[\mskip1.5mu \Conid{PdFloat}\;\Varid{delta},\Conid{PdFloat}\;\Varid{position}\mskip1.5mu])\mathrel{=}{}\<[E]%
\\
\>[B]{}\hsindent{4}{}\<[4]%
\>[4]{}\mathbf{let}{}\<[E]%
\\
\>[4]{}\hsindent{3}{}\<[7]%
\>[7]{}\Varid{osc}\mathbin{::}\Conid{Double}\to \Conid{Double}\to \Conid{Double}\to \Conid{Double}{}\<[E]%
\\
\>[4]{}\hsindent{3}{}\<[7]%
\>[7]{}\Varid{osc}\;\Varid{position}\;\Varid{delta}\;\Varid{idx}\mathrel{=}(\Varid{position}\mathbin{+}(\Varid{delta}\mathbin{*}\Varid{idx}))\mathbin{`\Varid{mod'}`}(\mathrm{2}\mathbin{*}\Varid{pi}){}\<[E]%
\\[\blanklineskip]%
\>[4]{}\hsindent{3}{}\<[7]%
\>[7]{}\Varid{output}\mathrel{=}\Varid{map}\;(\Conid{PdFloat}\mathbin{\circ}\Varid{sin}\mathbin{\circ}\Varid{osc}\;\Varid{position}\;\Varid{delta})\;[\mskip1.5mu \mathrm{0}\mathinner{\ldotp\ldotp}\mathrm{63}\mskip1.5mu]{}\<[E]%
\\
\>[4]{}\hsindent{3}{}\<[7]%
\>[7]{}\Varid{nextPosition}\mathrel{=}\Varid{osc}\;\Varid{position}\;\Varid{delta}\;\mathrm{64}{}\<[E]%
\\
\>[4]{}\hsindent{3}{}\<[7]%
\>[7]{}\Varid{mem'}\mathrel{=}[\mskip1.5mu \Conid{PdFloat}\;\Varid{delta},\Conid{PdFloat}\;\Varid{nextPosition}\mskip1.5mu]{}\<[E]%
\\
\>[B]{}\hsindent{4}{}\<[4]%
\>[4]{}\mathbf{in}{}\<[E]%
\\
\>[4]{}\hsindent{3}{}\<[7]%
\>[7]{}([\mskip1.5mu \Varid{output}\mskip1.5mu],\Varid{mem'}){}\<[E]%
\ColumnHook
\end{hscode}\resethooks

As described in Section~\ref{linemsg}, the \texttt{line\textasciitilde{}} object implements a
linear ramp over time. As in \texttt{osc\textasciitilde{}} we handle two cases: when the internal
memory of the object is empty, in which case we initialize it; and when it is
initialized with \ensuremath{\Varid{current}}, \ensuremath{\Varid{target}} and \ensuremath{\Varid{delta}} values. The
function varies linearly over time from \ensuremath{\Varid{current}} to \ensuremath{\Varid{target}}, after
which, it stays constant at \ensuremath{\Varid{target}}.

\begin{hscode}\SaveRestoreHook
\column{B}{@{}>{\hspre}l<{\hspost}@{}}%
\column{4}{@{}>{\hspre}l<{\hspost}@{}}%
\column{7}{@{}>{\hspre}l<{\hspost}@{}}%
\column{13}{@{}>{\hspre}l<{\hspost}@{}}%
\column{17}{@{}>{\hspre}l<{\hspost}@{}}%
\column{E}{@{}>{\hspre}l<{\hspost}@{}}%
\>[B]{}\Varid{performDsp}\;\Varid{obj}\mathord{@}(\Conid{PdObj}\;[\mskip1.5mu \Conid{PdSymbol}\;\text{\tt \char34 line\char126 \char34}\mskip1.5mu]\;\anonymous \;\anonymous )\;(\Conid{PdNodeState}\;\Varid{ins}\;[\mskip1.5mu \mskip1.5mu])\mathrel{=}{}\<[E]%
\\
\>[B]{}\hsindent{4}{}\<[4]%
\>[4]{}\Varid{performDsp}\;\Varid{obj}\;(\Conid{PdNodeState}\;\Varid{ins}\;[\mskip1.5mu \Conid{PdFloat}\;\mathrm{0},\Conid{PdFloat}\;\mathrm{0},\Conid{PdFloat}\;\mathrm{0}\mskip1.5mu]){}\<[E]%
\\[\blanklineskip]%
\>[B]{}\Varid{performDsp}\;{}\<[13]%
\>[13]{}(\Conid{PdObj}\;[\mskip1.5mu \Conid{PdSymbol}\;\text{\tt \char34 line\char126 \char34}\mskip1.5mu]\;\anonymous \;\anonymous )\;{}\<[E]%
\\
\>[13]{}(\Conid{PdNodeState}\;\Varid{ins}\;[\mskip1.5mu \Conid{PdFloat}\;\Varid{current},\Conid{PdFloat}\;\Varid{target},\Conid{PdFloat}\;\Varid{delta}\mskip1.5mu])\mathrel{=}{}\<[E]%
\\
\>[B]{}\hsindent{4}{}\<[4]%
\>[4]{}\mathbf{let}{}\<[E]%
\\
\>[4]{}\hsindent{3}{}\<[7]%
\>[7]{}\Varid{limiter}\mathrel{=}\mathbf{if}\;\Varid{delta}\mathbin{>}\mathrm{0}\;\mathbf{then}\;\Varid{min}\;\mathbf{else}\;\Varid{max}{}\<[E]%
\\
\>[4]{}\hsindent{3}{}\<[7]%
\>[7]{}\Varid{output}\mathrel{=}{}\<[17]%
\>[17]{}\Varid{map}\;\Conid{PdFloat}\mathbin{\$}\Varid{tail}\mathbin{\$}\Varid{take}\;\mathrm{65}{}\<[E]%
\\
\>[17]{}\mathbin{\$}\Varid{iterate}\;(\lambda \Varid{v}\to \Varid{limiter}\;\Varid{target}\;(\Varid{v}\mathbin{+}\Varid{delta}))\;\Varid{current}{}\<[E]%
\\
\>[4]{}\hsindent{3}{}\<[7]%
\>[7]{}\Varid{mem'}\mathrel{=}[\mskip1.5mu \Varid{last}\;\Varid{output},\Conid{PdFloat}\;\Varid{target},\Conid{PdFloat}\;\Varid{delta}\mskip1.5mu]{}\<[E]%
\\
\>[B]{}\hsindent{4}{}\<[4]%
\>[4]{}\mathbf{in}{}\<[E]%
\\
\>[4]{}\hsindent{3}{}\<[7]%
\>[7]{}([\mskip1.5mu \Varid{output}\mskip1.5mu],\Varid{mem'}){}\<[E]%
\ColumnHook
\end{hscode}\resethooks

The \text{\tt \char42{}\char126{}} object multiplies the data from inlets 0 and 1. It is used, for example,
to modify the amplitude of an audio wave.

\begin{hscode}\SaveRestoreHook
\column{B}{@{}>{\hspre}l<{\hspost}@{}}%
\column{4}{@{}>{\hspre}l<{\hspost}@{}}%
\column{7}{@{}>{\hspre}l<{\hspost}@{}}%
\column{E}{@{}>{\hspre}l<{\hspost}@{}}%
\>[B]{}\Varid{performDsp}\;(\Conid{PdObj}\;[\mskip1.5mu \Conid{PdSymbol}\;\text{\tt \char34 *\char126 \char34}\mskip1.5mu]\;\anonymous \;\anonymous )\;(\Conid{PdNodeState}\;\Varid{ins}\;[\mskip1.5mu \mskip1.5mu])\mathrel{=}{}\<[E]%
\\
\>[B]{}\hsindent{4}{}\<[4]%
\>[4]{}\mathbf{let}{}\<[E]%
\\
\>[4]{}\hsindent{3}{}\<[7]%
\>[7]{}\Varid{mult}\;(\Conid{PdFloat}\;\Varid{a})\;(\Conid{PdFloat}\;\Varid{b})\mathrel{=}\Conid{PdFloat}\;(\Varid{a}\mathbin{*}\Varid{b}){}\<[E]%
\\
\>[4]{}\hsindent{3}{}\<[7]%
\>[7]{}\Varid{output}\mathrel{=}\Varid{zipWith}\;\Varid{mult}\;(\Varid{index}\;\Varid{ins}\;\mathrm{0})\;(\Varid{index}\;\Varid{ins}\;\mathrm{1}){}\<[E]%
\\
\>[B]{}\hsindent{4}{}\<[4]%
\>[4]{}\mathbf{in}{}\<[E]%
\\
\>[4]{}\hsindent{3}{}\<[7]%
\>[7]{}([\mskip1.5mu \Varid{output}\mskip1.5mu],[\mskip1.5mu \mskip1.5mu]){}\<[E]%
\ColumnHook
\end{hscode}\resethooks

Finally, this is a default handler for \ensuremath{\Varid{performDsp}} that merely produces
a silent audio buffer.

\begin{hscode}\SaveRestoreHook
\column{B}{@{}>{\hspre}l<{\hspost}@{}}%
\column{4}{@{}>{\hspre}l<{\hspost}@{}}%
\column{E}{@{}>{\hspre}l<{\hspost}@{}}%
\>[B]{}\Varid{performDsp}\;\Varid{obj}\;\Varid{ns}\mathrel{=}{}\<[E]%
\\
\>[B]{}\hsindent{4}{}\<[4]%
\>[4]{}([\mskip1.5mu \Varid{toList}\mathbin{\$}\Varid{replicate}\;\mathrm{64}\mathbin{\$}\Conid{PdFloat}\;\mathrm{0.0}\mskip1.5mu],[\mskip1.5mu \mskip1.5mu]){}\<[E]%
\ColumnHook
\end{hscode}\resethooks

%END LYX TEXT
